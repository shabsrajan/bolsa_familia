
% Default to the notebook output style

    


% Inherit from the specified cell style.




    
\documentclass[11pt]{article}

    
    
    \usepackage[T1]{fontenc}
    % Nicer default font (+ math font) than Computer Modern for most use cases
    \usepackage{mathpazo}

    % Basic figure setup, for now with no caption control since it's done
    % automatically by Pandoc (which extracts ![](path) syntax from Markdown).
    \usepackage{graphicx}
    % We will generate all images so they have a width \maxwidth. This means
    % that they will get their normal width if they fit onto the page, but
    % are scaled down if they would overflow the margins.
    \makeatletter
    \def\maxwidth{\ifdim\Gin@nat@width>\linewidth\linewidth
    \else\Gin@nat@width\fi}
    \makeatother
    \let\Oldincludegraphics\includegraphics
    % Set max figure width to be 80% of text width, for now hardcoded.
    \renewcommand{\includegraphics}[1]{\Oldincludegraphics[width=.8\maxwidth]{#1}}
    % Ensure that by default, figures have no caption (until we provide a
    % proper Figure object with a Caption API and a way to capture that
    % in the conversion process - todo).
    \usepackage{caption}
    \DeclareCaptionLabelFormat{nolabel}{}
    \captionsetup{labelformat=nolabel}

    \usepackage{adjustbox} % Used to constrain images to a maximum size 
    \usepackage{xcolor} % Allow colors to be defined
    \usepackage{enumerate} % Needed for markdown enumerations to work
    \usepackage{geometry} % Used to adjust the document margins
    \usepackage{amsmath} % Equations
    \usepackage{amssymb} % Equations
    \usepackage{textcomp} % defines textquotesingle
    % Hack from http://tex.stackexchange.com/a/47451/13684:
    \AtBeginDocument{%
        \def\PYZsq{\textquotesingle}% Upright quotes in Pygmentized code
    }
    \usepackage{upquote} % Upright quotes for verbatim code
    \usepackage{eurosym} % defines \euro
    \usepackage[mathletters]{ucs} % Extended unicode (utf-8) support
    \usepackage[utf8x]{inputenc} % Allow utf-8 characters in the tex document
    \usepackage{fancyvrb} % verbatim replacement that allows latex
    \usepackage{grffile} % extends the file name processing of package graphics 
                         % to support a larger range 
    % The hyperref package gives us a pdf with properly built
    % internal navigation ('pdf bookmarks' for the table of contents,
    % internal cross-reference links, web links for URLs, etc.)
    \usepackage{hyperref}
    \usepackage{longtable} % longtable support required by pandoc >1.10
    \usepackage{booktabs}  % table support for pandoc > 1.12.2
    \usepackage[inline]{enumitem} % IRkernel/repr support (it uses the enumerate* environment)
    \usepackage[normalem]{ulem} % ulem is needed to support strikethroughs (\sout)
                                % normalem makes italics be italics, not underlines
    

    
    
    % Colors for the hyperref package
    \definecolor{urlcolor}{rgb}{0,.145,.698}
    \definecolor{linkcolor}{rgb}{.71,0.21,0.01}
    \definecolor{citecolor}{rgb}{.12,.54,.11}

    % ANSI colors
    \definecolor{ansi-black}{HTML}{3E424D}
    \definecolor{ansi-black-intense}{HTML}{282C36}
    \definecolor{ansi-red}{HTML}{E75C58}
    \definecolor{ansi-red-intense}{HTML}{B22B31}
    \definecolor{ansi-green}{HTML}{00A250}
    \definecolor{ansi-green-intense}{HTML}{007427}
    \definecolor{ansi-yellow}{HTML}{DDB62B}
    \definecolor{ansi-yellow-intense}{HTML}{B27D12}
    \definecolor{ansi-blue}{HTML}{208FFB}
    \definecolor{ansi-blue-intense}{HTML}{0065CA}
    \definecolor{ansi-magenta}{HTML}{D160C4}
    \definecolor{ansi-magenta-intense}{HTML}{A03196}
    \definecolor{ansi-cyan}{HTML}{60C6C8}
    \definecolor{ansi-cyan-intense}{HTML}{258F8F}
    \definecolor{ansi-white}{HTML}{C5C1B4}
    \definecolor{ansi-white-intense}{HTML}{A1A6B2}

    % commands and environments needed by pandoc snippets
    % extracted from the output of `pandoc -s`
    \providecommand{\tightlist}{%
      \setlength{\itemsep}{0pt}\setlength{\parskip}{0pt}}
    \DefineVerbatimEnvironment{Highlighting}{Verbatim}{commandchars=\\\{\}}
    % Add ',fontsize=\small' for more characters per line
    \newenvironment{Shaded}{}{}
    \newcommand{\KeywordTok}[1]{\textcolor[rgb]{0.00,0.44,0.13}{\textbf{{#1}}}}
    \newcommand{\DataTypeTok}[1]{\textcolor[rgb]{0.56,0.13,0.00}{{#1}}}
    \newcommand{\DecValTok}[1]{\textcolor[rgb]{0.25,0.63,0.44}{{#1}}}
    \newcommand{\BaseNTok}[1]{\textcolor[rgb]{0.25,0.63,0.44}{{#1}}}
    \newcommand{\FloatTok}[1]{\textcolor[rgb]{0.25,0.63,0.44}{{#1}}}
    \newcommand{\CharTok}[1]{\textcolor[rgb]{0.25,0.44,0.63}{{#1}}}
    \newcommand{\StringTok}[1]{\textcolor[rgb]{0.25,0.44,0.63}{{#1}}}
    \newcommand{\CommentTok}[1]{\textcolor[rgb]{0.38,0.63,0.69}{\textit{{#1}}}}
    \newcommand{\OtherTok}[1]{\textcolor[rgb]{0.00,0.44,0.13}{{#1}}}
    \newcommand{\AlertTok}[1]{\textcolor[rgb]{1.00,0.00,0.00}{\textbf{{#1}}}}
    \newcommand{\FunctionTok}[1]{\textcolor[rgb]{0.02,0.16,0.49}{{#1}}}
    \newcommand{\RegionMarkerTok}[1]{{#1}}
    \newcommand{\ErrorTok}[1]{\textcolor[rgb]{1.00,0.00,0.00}{\textbf{{#1}}}}
    \newcommand{\NormalTok}[1]{{#1}}
    
    % Additional commands for more recent versions of Pandoc
    \newcommand{\ConstantTok}[1]{\textcolor[rgb]{0.53,0.00,0.00}{{#1}}}
    \newcommand{\SpecialCharTok}[1]{\textcolor[rgb]{0.25,0.44,0.63}{{#1}}}
    \newcommand{\VerbatimStringTok}[1]{\textcolor[rgb]{0.25,0.44,0.63}{{#1}}}
    \newcommand{\SpecialStringTok}[1]{\textcolor[rgb]{0.73,0.40,0.53}{{#1}}}
    \newcommand{\ImportTok}[1]{{#1}}
    \newcommand{\DocumentationTok}[1]{\textcolor[rgb]{0.73,0.13,0.13}{\textit{{#1}}}}
    \newcommand{\AnnotationTok}[1]{\textcolor[rgb]{0.38,0.63,0.69}{\textbf{\textit{{#1}}}}}
    \newcommand{\CommentVarTok}[1]{\textcolor[rgb]{0.38,0.63,0.69}{\textbf{\textit{{#1}}}}}
    \newcommand{\VariableTok}[1]{\textcolor[rgb]{0.10,0.09,0.49}{{#1}}}
    \newcommand{\ControlFlowTok}[1]{\textcolor[rgb]{0.00,0.44,0.13}{\textbf{{#1}}}}
    \newcommand{\OperatorTok}[1]{\textcolor[rgb]{0.40,0.40,0.40}{{#1}}}
    \newcommand{\BuiltInTok}[1]{{#1}}
    \newcommand{\ExtensionTok}[1]{{#1}}
    \newcommand{\PreprocessorTok}[1]{\textcolor[rgb]{0.74,0.48,0.00}{{#1}}}
    \newcommand{\AttributeTok}[1]{\textcolor[rgb]{0.49,0.56,0.16}{{#1}}}
    \newcommand{\InformationTok}[1]{\textcolor[rgb]{0.38,0.63,0.69}{\textbf{\textit{{#1}}}}}
    \newcommand{\WarningTok}[1]{\textcolor[rgb]{0.38,0.63,0.69}{\textbf{\textit{{#1}}}}}
    
    
    % Define a nice break command that doesn't care if a line doesn't already
    % exist.
    \def\br{\hspace*{\fill} \\* }
    % Math Jax compatability definitions
    \def\gt{>}
    \def\lt{<}
    % Document parameters
    \title{Predicting-no-shows}
    
    
    

    % Pygments definitions
    
\makeatletter
\def\PY@reset{\let\PY@it=\relax \let\PY@bf=\relax%
    \let\PY@ul=\relax \let\PY@tc=\relax%
    \let\PY@bc=\relax \let\PY@ff=\relax}
\def\PY@tok#1{\csname PY@tok@#1\endcsname}
\def\PY@toks#1+{\ifx\relax#1\empty\else%
    \PY@tok{#1}\expandafter\PY@toks\fi}
\def\PY@do#1{\PY@bc{\PY@tc{\PY@ul{%
    \PY@it{\PY@bf{\PY@ff{#1}}}}}}}
\def\PY#1#2{\PY@reset\PY@toks#1+\relax+\PY@do{#2}}

\expandafter\def\csname PY@tok@w\endcsname{\def\PY@tc##1{\textcolor[rgb]{0.73,0.73,0.73}{##1}}}
\expandafter\def\csname PY@tok@c\endcsname{\let\PY@it=\textit\def\PY@tc##1{\textcolor[rgb]{0.25,0.50,0.50}{##1}}}
\expandafter\def\csname PY@tok@cp\endcsname{\def\PY@tc##1{\textcolor[rgb]{0.74,0.48,0.00}{##1}}}
\expandafter\def\csname PY@tok@k\endcsname{\let\PY@bf=\textbf\def\PY@tc##1{\textcolor[rgb]{0.00,0.50,0.00}{##1}}}
\expandafter\def\csname PY@tok@kp\endcsname{\def\PY@tc##1{\textcolor[rgb]{0.00,0.50,0.00}{##1}}}
\expandafter\def\csname PY@tok@kt\endcsname{\def\PY@tc##1{\textcolor[rgb]{0.69,0.00,0.25}{##1}}}
\expandafter\def\csname PY@tok@o\endcsname{\def\PY@tc##1{\textcolor[rgb]{0.40,0.40,0.40}{##1}}}
\expandafter\def\csname PY@tok@ow\endcsname{\let\PY@bf=\textbf\def\PY@tc##1{\textcolor[rgb]{0.67,0.13,1.00}{##1}}}
\expandafter\def\csname PY@tok@nb\endcsname{\def\PY@tc##1{\textcolor[rgb]{0.00,0.50,0.00}{##1}}}
\expandafter\def\csname PY@tok@nf\endcsname{\def\PY@tc##1{\textcolor[rgb]{0.00,0.00,1.00}{##1}}}
\expandafter\def\csname PY@tok@nc\endcsname{\let\PY@bf=\textbf\def\PY@tc##1{\textcolor[rgb]{0.00,0.00,1.00}{##1}}}
\expandafter\def\csname PY@tok@nn\endcsname{\let\PY@bf=\textbf\def\PY@tc##1{\textcolor[rgb]{0.00,0.00,1.00}{##1}}}
\expandafter\def\csname PY@tok@ne\endcsname{\let\PY@bf=\textbf\def\PY@tc##1{\textcolor[rgb]{0.82,0.25,0.23}{##1}}}
\expandafter\def\csname PY@tok@nv\endcsname{\def\PY@tc##1{\textcolor[rgb]{0.10,0.09,0.49}{##1}}}
\expandafter\def\csname PY@tok@no\endcsname{\def\PY@tc##1{\textcolor[rgb]{0.53,0.00,0.00}{##1}}}
\expandafter\def\csname PY@tok@nl\endcsname{\def\PY@tc##1{\textcolor[rgb]{0.63,0.63,0.00}{##1}}}
\expandafter\def\csname PY@tok@ni\endcsname{\let\PY@bf=\textbf\def\PY@tc##1{\textcolor[rgb]{0.60,0.60,0.60}{##1}}}
\expandafter\def\csname PY@tok@na\endcsname{\def\PY@tc##1{\textcolor[rgb]{0.49,0.56,0.16}{##1}}}
\expandafter\def\csname PY@tok@nt\endcsname{\let\PY@bf=\textbf\def\PY@tc##1{\textcolor[rgb]{0.00,0.50,0.00}{##1}}}
\expandafter\def\csname PY@tok@nd\endcsname{\def\PY@tc##1{\textcolor[rgb]{0.67,0.13,1.00}{##1}}}
\expandafter\def\csname PY@tok@s\endcsname{\def\PY@tc##1{\textcolor[rgb]{0.73,0.13,0.13}{##1}}}
\expandafter\def\csname PY@tok@sd\endcsname{\let\PY@it=\textit\def\PY@tc##1{\textcolor[rgb]{0.73,0.13,0.13}{##1}}}
\expandafter\def\csname PY@tok@si\endcsname{\let\PY@bf=\textbf\def\PY@tc##1{\textcolor[rgb]{0.73,0.40,0.53}{##1}}}
\expandafter\def\csname PY@tok@se\endcsname{\let\PY@bf=\textbf\def\PY@tc##1{\textcolor[rgb]{0.73,0.40,0.13}{##1}}}
\expandafter\def\csname PY@tok@sr\endcsname{\def\PY@tc##1{\textcolor[rgb]{0.73,0.40,0.53}{##1}}}
\expandafter\def\csname PY@tok@ss\endcsname{\def\PY@tc##1{\textcolor[rgb]{0.10,0.09,0.49}{##1}}}
\expandafter\def\csname PY@tok@sx\endcsname{\def\PY@tc##1{\textcolor[rgb]{0.00,0.50,0.00}{##1}}}
\expandafter\def\csname PY@tok@m\endcsname{\def\PY@tc##1{\textcolor[rgb]{0.40,0.40,0.40}{##1}}}
\expandafter\def\csname PY@tok@gh\endcsname{\let\PY@bf=\textbf\def\PY@tc##1{\textcolor[rgb]{0.00,0.00,0.50}{##1}}}
\expandafter\def\csname PY@tok@gu\endcsname{\let\PY@bf=\textbf\def\PY@tc##1{\textcolor[rgb]{0.50,0.00,0.50}{##1}}}
\expandafter\def\csname PY@tok@gd\endcsname{\def\PY@tc##1{\textcolor[rgb]{0.63,0.00,0.00}{##1}}}
\expandafter\def\csname PY@tok@gi\endcsname{\def\PY@tc##1{\textcolor[rgb]{0.00,0.63,0.00}{##1}}}
\expandafter\def\csname PY@tok@gr\endcsname{\def\PY@tc##1{\textcolor[rgb]{1.00,0.00,0.00}{##1}}}
\expandafter\def\csname PY@tok@ge\endcsname{\let\PY@it=\textit}
\expandafter\def\csname PY@tok@gs\endcsname{\let\PY@bf=\textbf}
\expandafter\def\csname PY@tok@gp\endcsname{\let\PY@bf=\textbf\def\PY@tc##1{\textcolor[rgb]{0.00,0.00,0.50}{##1}}}
\expandafter\def\csname PY@tok@go\endcsname{\def\PY@tc##1{\textcolor[rgb]{0.53,0.53,0.53}{##1}}}
\expandafter\def\csname PY@tok@gt\endcsname{\def\PY@tc##1{\textcolor[rgb]{0.00,0.27,0.87}{##1}}}
\expandafter\def\csname PY@tok@err\endcsname{\def\PY@bc##1{\setlength{\fboxsep}{0pt}\fcolorbox[rgb]{1.00,0.00,0.00}{1,1,1}{\strut ##1}}}
\expandafter\def\csname PY@tok@kc\endcsname{\let\PY@bf=\textbf\def\PY@tc##1{\textcolor[rgb]{0.00,0.50,0.00}{##1}}}
\expandafter\def\csname PY@tok@kd\endcsname{\let\PY@bf=\textbf\def\PY@tc##1{\textcolor[rgb]{0.00,0.50,0.00}{##1}}}
\expandafter\def\csname PY@tok@kn\endcsname{\let\PY@bf=\textbf\def\PY@tc##1{\textcolor[rgb]{0.00,0.50,0.00}{##1}}}
\expandafter\def\csname PY@tok@kr\endcsname{\let\PY@bf=\textbf\def\PY@tc##1{\textcolor[rgb]{0.00,0.50,0.00}{##1}}}
\expandafter\def\csname PY@tok@bp\endcsname{\def\PY@tc##1{\textcolor[rgb]{0.00,0.50,0.00}{##1}}}
\expandafter\def\csname PY@tok@fm\endcsname{\def\PY@tc##1{\textcolor[rgb]{0.00,0.00,1.00}{##1}}}
\expandafter\def\csname PY@tok@vc\endcsname{\def\PY@tc##1{\textcolor[rgb]{0.10,0.09,0.49}{##1}}}
\expandafter\def\csname PY@tok@vg\endcsname{\def\PY@tc##1{\textcolor[rgb]{0.10,0.09,0.49}{##1}}}
\expandafter\def\csname PY@tok@vi\endcsname{\def\PY@tc##1{\textcolor[rgb]{0.10,0.09,0.49}{##1}}}
\expandafter\def\csname PY@tok@vm\endcsname{\def\PY@tc##1{\textcolor[rgb]{0.10,0.09,0.49}{##1}}}
\expandafter\def\csname PY@tok@sa\endcsname{\def\PY@tc##1{\textcolor[rgb]{0.73,0.13,0.13}{##1}}}
\expandafter\def\csname PY@tok@sb\endcsname{\def\PY@tc##1{\textcolor[rgb]{0.73,0.13,0.13}{##1}}}
\expandafter\def\csname PY@tok@sc\endcsname{\def\PY@tc##1{\textcolor[rgb]{0.73,0.13,0.13}{##1}}}
\expandafter\def\csname PY@tok@dl\endcsname{\def\PY@tc##1{\textcolor[rgb]{0.73,0.13,0.13}{##1}}}
\expandafter\def\csname PY@tok@s2\endcsname{\def\PY@tc##1{\textcolor[rgb]{0.73,0.13,0.13}{##1}}}
\expandafter\def\csname PY@tok@sh\endcsname{\def\PY@tc##1{\textcolor[rgb]{0.73,0.13,0.13}{##1}}}
\expandafter\def\csname PY@tok@s1\endcsname{\def\PY@tc##1{\textcolor[rgb]{0.73,0.13,0.13}{##1}}}
\expandafter\def\csname PY@tok@mb\endcsname{\def\PY@tc##1{\textcolor[rgb]{0.40,0.40,0.40}{##1}}}
\expandafter\def\csname PY@tok@mf\endcsname{\def\PY@tc##1{\textcolor[rgb]{0.40,0.40,0.40}{##1}}}
\expandafter\def\csname PY@tok@mh\endcsname{\def\PY@tc##1{\textcolor[rgb]{0.40,0.40,0.40}{##1}}}
\expandafter\def\csname PY@tok@mi\endcsname{\def\PY@tc##1{\textcolor[rgb]{0.40,0.40,0.40}{##1}}}
\expandafter\def\csname PY@tok@il\endcsname{\def\PY@tc##1{\textcolor[rgb]{0.40,0.40,0.40}{##1}}}
\expandafter\def\csname PY@tok@mo\endcsname{\def\PY@tc##1{\textcolor[rgb]{0.40,0.40,0.40}{##1}}}
\expandafter\def\csname PY@tok@ch\endcsname{\let\PY@it=\textit\def\PY@tc##1{\textcolor[rgb]{0.25,0.50,0.50}{##1}}}
\expandafter\def\csname PY@tok@cm\endcsname{\let\PY@it=\textit\def\PY@tc##1{\textcolor[rgb]{0.25,0.50,0.50}{##1}}}
\expandafter\def\csname PY@tok@cpf\endcsname{\let\PY@it=\textit\def\PY@tc##1{\textcolor[rgb]{0.25,0.50,0.50}{##1}}}
\expandafter\def\csname PY@tok@c1\endcsname{\let\PY@it=\textit\def\PY@tc##1{\textcolor[rgb]{0.25,0.50,0.50}{##1}}}
\expandafter\def\csname PY@tok@cs\endcsname{\let\PY@it=\textit\def\PY@tc##1{\textcolor[rgb]{0.25,0.50,0.50}{##1}}}

\def\PYZbs{\char`\\}
\def\PYZus{\char`\_}
\def\PYZob{\char`\{}
\def\PYZcb{\char`\}}
\def\PYZca{\char`\^}
\def\PYZam{\char`\&}
\def\PYZlt{\char`\<}
\def\PYZgt{\char`\>}
\def\PYZsh{\char`\#}
\def\PYZpc{\char`\%}
\def\PYZdl{\char`\$}
\def\PYZhy{\char`\-}
\def\PYZsq{\char`\'}
\def\PYZdq{\char`\"}
\def\PYZti{\char`\~}
% for compatibility with earlier versions
\def\PYZat{@}
\def\PYZlb{[}
\def\PYZrb{]}
\makeatother


    % Exact colors from NB
    \definecolor{incolor}{rgb}{0.0, 0.0, 0.5}
    \definecolor{outcolor}{rgb}{0.545, 0.0, 0.0}



    
    % Prevent overflowing lines due to hard-to-break entities
    \sloppy 
    % Setup hyperref package
    \hypersetup{
      breaklinks=true,  % so long urls are correctly broken across lines
      colorlinks=true,
      urlcolor=urlcolor,
      linkcolor=linkcolor,
      citecolor=citecolor,
      }
    % Slightly bigger margins than the latex defaults
    
    \geometry{verbose,tmargin=1in,bmargin=1in,lmargin=1in,rmargin=1in}
    
    

    \begin{document}
    
    
    \maketitle
    
    

    
    \section{Project: Predicting no
shows}\label{project-predicting-no-shows}

Author: Shabs Rajasekharan. Report created as part fo the Udacity
Nanodegree program. Dataset used: No-Show Appointments dataset.

\subsection{Table of Contents}\label{table-of-contents}

Introduction

Data Wrangling

Exploratory Data Analysis

Conclusions

     \#\# Introduction

This dataset collects information from 100k medical appointments in
Brazil and is focused on the question of whether or not patients show up
for their appointment. Its original source can be found here:
https://www.kaggle.com/joniarroba/noshowappointments

Although there is ample contextual information publicly available about
the larger program from which this dataset was pulled (see context
below), it is not clear what came first: the question or the data? It is
very likely that it was the data. This is highlighted here because the
main question (originally posted on Kaggle) may have been planted
without considering the data collected to answer it.

\subsection{Main research question: What factors are important for us to
know in order to predict if a patient will show up for their scheduled
appointment?}\label{main-research-question-what-factors-are-important-for-us-to-know-in-order-to-predict-if-a-patient-will-show-up-for-their-scheduled-appointment}

    \subsection{Context}\label{context}

The observations in the dataset are part of the Bolsa Família social
welfare program of the Brazilian government. Bolsa Família provides
financial aid (income cash transfer) to poor Brazilian families, unders
specific conditions. The program attempts to both reduce short-term
poverty by direct cash transfers and fight long-term poverty by
increasing human capital among the poor through conditional cash
transfers. About 12 million Brazilian families receive funds from Bolsa
Família (in 2011, 26\% of the population were covered) and it has been
mentioned as one factor contributing to the reduction of poverty in
Brazil, which fell 27.7\% during the first term in the Luiz Inácio Lula
da Silva administration (2003 - 2011).

Further information on the Bolsa Familia program can be found here:
https://factsreports.revues.org/1560

    \subsection{About the data file and
version}\label{about-the-data-file-and-version}

A note is made here about the version of this data. It has been
manipulated and pre-processed several times by the public on Kaggle.
This analysis is based on version 6.

    \begin{Verbatim}[commandchars=\\\{\}]
{\color{incolor}In [{\color{incolor}1}]:} \PY{c+c1}{\PYZsh{} Packages I plan to use}
        \PY{o}{\PYZpc{}} \PY{n}{matplotlib} \PY{n}{inline} 
        
        \PY{k+kn}{import} \PY{n+nn}{pandas} \PY{k}{as} \PY{n+nn}{pd}
        \PY{k+kn}{import} \PY{n+nn}{numpy} \PY{k}{as} \PY{n+nn}{np}
        \PY{k+kn}{import} \PY{n+nn}{matplotlib}\PY{n+nn}{.}\PY{n+nn}{pyplot} \PY{k}{as} \PY{n+nn}{plt}
        \PY{k+kn}{import} \PY{n+nn}{datetime}
        \PY{k+kn}{import} \PY{n+nn}{seaborn} \PY{k}{as} \PY{n+nn}{sns}
        
        \PY{c+c1}{\PYZsh{}I want to display all the code in a shell, not just the last output}
        \PY{k+kn}{from} \PY{n+nn}{IPython}\PY{n+nn}{.}\PY{n+nn}{core}\PY{n+nn}{.}\PY{n+nn}{interactiveshell} \PY{k}{import} \PY{n}{InteractiveShell}
        \PY{n}{InteractiveShell}\PY{o}{.}\PY{n}{ast\PYZus{}node\PYZus{}interactivity} \PY{o}{=} \PY{l+s+s2}{\PYZdq{}}\PY{l+s+s2}{all}\PY{l+s+s2}{\PYZdq{}}
\end{Verbatim}


     \#\# Data structure and general properties

    \begin{Verbatim}[commandchars=\\\{\}]
{\color{incolor}In [{\color{incolor}2}]:} \PY{c+c1}{\PYZsh{} Load the data}
        \PY{n}{df} \PY{o}{=} \PY{n}{pd}\PY{o}{.}\PY{n}{read\PYZus{}csv}\PY{p}{(}\PY{l+s+s2}{\PYZdq{}}\PY{l+s+s2}{data/KaggleV2\PYZhy{}May\PYZhy{}2016.csv}\PY{l+s+s2}{\PYZdq{}}\PY{p}{)}
        \PY{n}{df}\PY{o}{.}\PY{n}{shape}
\end{Verbatim}


\begin{Verbatim}[commandchars=\\\{\}]
{\color{outcolor}Out[{\color{outcolor}2}]:} (110527, 14)
\end{Verbatim}
            
    \begin{Verbatim}[commandchars=\\\{\}]
{\color{incolor}In [{\color{incolor}3}]:} \PY{c+c1}{\PYZsh{}lets have a look at the first and last lines of data and a random sample}
        \PY{n}{df}\PY{o}{.}\PY{n}{head}\PY{p}{(}\PY{p}{)}
        \PY{n}{df}\PY{o}{.}\PY{n}{tail}\PY{p}{(}\PY{p}{)}
        \PY{n}{df}\PY{o}{.}\PY{n}{sample}\PY{p}{(}\PY{l+m+mi}{10}\PY{p}{)}
\end{Verbatim}


\begin{Verbatim}[commandchars=\\\{\}]
{\color{outcolor}Out[{\color{outcolor}3}]:}       PatientId  AppointmentID Gender          ScheduledDay  \textbackslash{}
        0  2.987250e+13        5642903      F  2016-04-29T18:38:08Z   
        1  5.589978e+14        5642503      M  2016-04-29T16:08:27Z   
        2  4.262962e+12        5642549      F  2016-04-29T16:19:04Z   
        3  8.679512e+11        5642828      F  2016-04-29T17:29:31Z   
        4  8.841186e+12        5642494      F  2016-04-29T16:07:23Z   
        
                 AppointmentDay  Age      Neighbourhood  Scholarship  Hipertension  \textbackslash{}
        0  2016-04-29T00:00:00Z   62    JARDIM DA PENHA            0             1   
        1  2016-04-29T00:00:00Z   56    JARDIM DA PENHA            0             0   
        2  2016-04-29T00:00:00Z   62      MATA DA PRAIA            0             0   
        3  2016-04-29T00:00:00Z    8  PONTAL DE CAMBURI            0             0   
        4  2016-04-29T00:00:00Z   56    JARDIM DA PENHA            0             1   
        
           Diabetes  Alcoholism  Handcap  SMS\_received No-show  
        0         0           0        0             0      No  
        1         0           0        0             0      No  
        2         0           0        0             0      No  
        3         0           0        0             0      No  
        4         1           0        0             0      No  
\end{Verbatim}
            
\begin{Verbatim}[commandchars=\\\{\}]
{\color{outcolor}Out[{\color{outcolor}3}]:}            PatientId  AppointmentID Gender          ScheduledDay  \textbackslash{}
        110522  2.572134e+12        5651768      F  2016-05-03T09:15:35Z   
        110523  3.596266e+12        5650093      F  2016-05-03T07:27:33Z   
        110524  1.557663e+13        5630692      F  2016-04-27T16:03:52Z   
        110525  9.213493e+13        5630323      F  2016-04-27T15:09:23Z   
        110526  3.775115e+14        5629448      F  2016-04-27T13:30:56Z   
        
                      AppointmentDay  Age Neighbourhood  Scholarship  Hipertension  \textbackslash{}
        110522  2016-06-07T00:00:00Z   56   MARIA ORTIZ            0             0   
        110523  2016-06-07T00:00:00Z   51   MARIA ORTIZ            0             0   
        110524  2016-06-07T00:00:00Z   21   MARIA ORTIZ            0             0   
        110525  2016-06-07T00:00:00Z   38   MARIA ORTIZ            0             0   
        110526  2016-06-07T00:00:00Z   54   MARIA ORTIZ            0             0   
        
                Diabetes  Alcoholism  Handcap  SMS\_received No-show  
        110522         0           0        0             1      No  
        110523         0           0        0             1      No  
        110524         0           0        0             1      No  
        110525         0           0        0             1      No  
        110526         0           0        0             1      No  
\end{Verbatim}
            
\begin{Verbatim}[commandchars=\\\{\}]
{\color{outcolor}Out[{\color{outcolor}3}]:}            PatientId  AppointmentID Gender          ScheduledDay  \textbackslash{}
        85010   8.377838e+13        5743569      F  2016-05-30T09:09:06Z   
        74616   6.579565e+13        5706809      F  2016-05-17T08:52:03Z   
        25590   4.144474e+11        5644597      F  2016-05-02T08:52:09Z   
        3966    1.436627e+11        5561975      F  2016-04-08T11:48:11Z   
        84019   7.859988e+12        5705407      F  2016-05-17T07:10:20Z   
        105218  3.897142e+14        5740691      F  2016-05-25T15:52:04Z   
        36699   8.836374e+13        5626175      F  2016-04-27T07:56:19Z   
        86921   5.471942e+14        5724295      F  2016-05-20T09:19:10Z   
        76042   5.431535e+12        5637850      F  2016-04-29T07:26:44Z   
        110213  7.246521e+12        5627640      F  2016-04-27T09:44:00Z   
        
                      AppointmentDay  Age   Neighbourhood  Scholarship  Hipertension  \textbackslash{}
        85010   2016-06-01T00:00:00Z   12  BARRO VERMELHO            0             0   
        74616   2016-05-19T00:00:00Z   63  JARDIM CAMBURI            0             0   
        25590   2016-05-10T00:00:00Z    3       REPÚBLICA            0             0   
        3966    2016-05-04T00:00:00Z   63     MARIA ORTIZ            0             0   
        84019   2016-05-17T00:00:00Z   79      TABUAZEIRO            0             1   
        105218  2016-06-01T00:00:00Z   67      MONTE BELO            0             0   
        36699   2016-05-03T00:00:00Z   32     RESISTÊNCIA            0             0   
        86921   2016-06-01T00:00:00Z    8    PRAIA DO SUÁ            1             0   
        76042   2016-05-02T00:00:00Z   25       SÃO PEDRO            0             0   
        110213  2016-06-02T00:00:00Z   51     MARIA ORTIZ            0             0   
        
                Diabetes  Alcoholism  Handcap  SMS\_received No-show  
        85010          0           0        0             0      No  
        74616          0           0        0             0      No  
        25590          0           0        0             1      No  
        3966           0           0        0             1      No  
        84019          1           0        0             0      No  
        105218         0           0        0             1      No  
        36699          0           0        0             1      No  
        86921          0           0        0             1      No  
        76042          0           0        0             1      No  
        110213         0           0        0             1      No  
\end{Verbatim}
            
    From public descriptions of the data, we know that `ScheduledDay' tells
us on what day the patient set up their appointment. `Neighborhood'
indicates the location of the hospital. `Scholarship' indicates whether
or not the patient is enrolled in Brasilian welfare program Bolsa
Família.

The last column: it says `No' if the patient showed up to their
appointment, and `Yes' if they did not show up.

    \begin{Verbatim}[commandchars=\\\{\}]
{\color{incolor}In [{\color{incolor}6}]:} \PY{c+c1}{\PYZsh{} Are there any missing or NaN values? }
        \PY{n}{df}\PY{o}{.}\PY{n}{describe}\PY{p}{(}\PY{p}{)}
        \PY{n}{df}\PY{o}{.}\PY{n}{info}\PY{p}{(}\PY{p}{)}
        \PY{n}{df}\PY{p}{[}\PY{l+s+s1}{\PYZsq{}}\PY{l+s+s1}{PatientId}\PY{l+s+s1}{\PYZsq{}}\PY{p}{]}\PY{o}{.}\PY{n}{count}\PY{p}{(}\PY{p}{)}
        \PY{n}{df}\PY{o}{.}\PY{n}{isnull}\PY{p}{(}\PY{p}{)}\PY{o}{.}\PY{n}{sum}\PY{p}{(}\PY{p}{)}
\end{Verbatim}


\begin{Verbatim}[commandchars=\\\{\}]
{\color{outcolor}Out[{\color{outcolor}6}]:}           PatientId  AppointmentID            Age    Scholarship  \textbackslash{}
        count  1.105270e+05   1.105270e+05  110527.000000  110527.000000   
        mean   1.474963e+14   5.675305e+06      37.088874       0.098266   
        std    2.560949e+14   7.129575e+04      23.110205       0.297675   
        min    3.921784e+04   5.030230e+06      -1.000000       0.000000   
        25\%    4.172614e+12   5.640286e+06      18.000000       0.000000   
        50\%    3.173184e+13   5.680573e+06      37.000000       0.000000   
        75\%    9.439172e+13   5.725524e+06      55.000000       0.000000   
        max    9.999816e+14   5.790484e+06     115.000000       1.000000   
        
                Hipertension       Diabetes     Alcoholism        Handcap  \textbackslash{}
        count  110527.000000  110527.000000  110527.000000  110527.000000   
        mean        0.197246       0.071865       0.030400       0.022248   
        std         0.397921       0.258265       0.171686       0.161543   
        min         0.000000       0.000000       0.000000       0.000000   
        25\%         0.000000       0.000000       0.000000       0.000000   
        50\%         0.000000       0.000000       0.000000       0.000000   
        75\%         0.000000       0.000000       0.000000       0.000000   
        max         1.000000       1.000000       1.000000       4.000000   
        
                SMS\_received  
        count  110527.000000  
        mean        0.321026  
        std         0.466873  
        min         0.000000  
        25\%         0.000000  
        50\%         0.000000  
        75\%         1.000000  
        max         1.000000  
\end{Verbatim}
            
    \begin{Verbatim}[commandchars=\\\{\}]
<class 'pandas.core.frame.DataFrame'>
RangeIndex: 110527 entries, 0 to 110526
Data columns (total 14 columns):
PatientId         110527 non-null float64
AppointmentID     110527 non-null int64
Gender            110527 non-null object
ScheduledDay      110527 non-null object
AppointmentDay    110527 non-null object
Age               110527 non-null int64
Neighbourhood     110527 non-null object
Scholarship       110527 non-null int64
Hipertension      110527 non-null int64
Diabetes          110527 non-null int64
Alcoholism        110527 non-null int64
Handcap           110527 non-null int64
SMS\_received      110527 non-null int64
No-show           110527 non-null object
dtypes: float64(1), int64(8), object(5)
memory usage: 11.8+ MB

    \end{Verbatim}

\begin{Verbatim}[commandchars=\\\{\}]
{\color{outcolor}Out[{\color{outcolor}6}]:} 110527
\end{Verbatim}
            
\begin{Verbatim}[commandchars=\\\{\}]
{\color{outcolor}Out[{\color{outcolor}6}]:} PatientId         0
        AppointmentID     0
        Gender            0
        ScheduledDay      0
        AppointmentDay    0
        Age               0
        Neighbourhood     0
        Scholarship       0
        Hipertension      0
        Diabetes          0
        Alcoholism        0
        Handcap           0
        SMS\_received      0
        No-show           0
        dtype: int64
\end{Verbatim}
            
    According to the following blocks of results above, there are no NaN or
missing values. Here the count value is the same as that in the shape
output, meanign there are no missing values and also there are no NaN
values according to the isnull function output.

    \subsubsection{Data Cleaning: data types, outliers and other
anomalies}\label{data-cleaning-data-types-outliers-and-other-anomalies}

    \begin{Verbatim}[commandchars=\\\{\}]
{\color{incolor}In [{\color{incolor}7}]:} \PY{c+c1}{\PYZsh{} In the previous section, it was clear that there may be some strings that should be in other formats}
        \PY{c+c1}{\PYZsh{} let\PYZsq{}s check out the data types once more}
        \PY{n}{df}\PY{o}{.}\PY{n}{dtypes}
\end{Verbatim}


\begin{Verbatim}[commandchars=\\\{\}]
{\color{outcolor}Out[{\color{outcolor}7}]:} PatientId         float64
        AppointmentID       int64
        Gender             object
        ScheduledDay       object
        AppointmentDay     object
        Age                 int64
        Neighbourhood      object
        Scholarship         int64
        Hipertension        int64
        Diabetes            int64
        Alcoholism          int64
        Handcap             int64
        SMS\_received        int64
        No-show            object
        dtype: object
\end{Verbatim}
            
    \begin{Verbatim}[commandchars=\\\{\}]
{\color{incolor}In [{\color{incolor}8}]:} \PY{c+c1}{\PYZsh{} Patient ID would be better as a number rather than an exponential. }
        \PY{c+c1}{\PYZsh{} This will not affect analysis but it will make reading the data easier}
        \PY{n}{df}\PY{p}{[}\PY{l+s+s1}{\PYZsq{}}\PY{l+s+s1}{PatientId}\PY{l+s+s1}{\PYZsq{}}\PY{p}{]} \PY{o}{=} \PY{n}{df}\PY{p}{[}\PY{l+s+s1}{\PYZsq{}}\PY{l+s+s1}{PatientId}\PY{l+s+s1}{\PYZsq{}}\PY{p}{]}\PY{o}{.}\PY{n}{astype}\PY{p}{(}\PY{l+s+s1}{\PYZsq{}}\PY{l+s+s1}{int64}\PY{l+s+s1}{\PYZsq{}}\PY{p}{)}
        \PY{n}{df}\PY{p}{[}\PY{l+s+s1}{\PYZsq{}}\PY{l+s+s1}{PatientId}\PY{l+s+s1}{\PYZsq{}}\PY{p}{]}\PY{o}{.}\PY{n}{dtype}
        \PY{n}{df}\PY{o}{.}\PY{n}{head}\PY{p}{(}\PY{p}{)}
\end{Verbatim}


\begin{Verbatim}[commandchars=\\\{\}]
{\color{outcolor}Out[{\color{outcolor}8}]:} dtype('int64')
\end{Verbatim}
            
\begin{Verbatim}[commandchars=\\\{\}]
{\color{outcolor}Out[{\color{outcolor}8}]:}          PatientId  AppointmentID Gender          ScheduledDay  \textbackslash{}
        0   29872499824296        5642903      F  2016-04-29T18:38:08Z   
        1  558997776694438        5642503      M  2016-04-29T16:08:27Z   
        2    4262962299951        5642549      F  2016-04-29T16:19:04Z   
        3     867951213174        5642828      F  2016-04-29T17:29:31Z   
        4    8841186448183        5642494      F  2016-04-29T16:07:23Z   
        
                 AppointmentDay  Age      Neighbourhood  Scholarship  Hipertension  \textbackslash{}
        0  2016-04-29T00:00:00Z   62    JARDIM DA PENHA            0             1   
        1  2016-04-29T00:00:00Z   56    JARDIM DA PENHA            0             0   
        2  2016-04-29T00:00:00Z   62      MATA DA PRAIA            0             0   
        3  2016-04-29T00:00:00Z    8  PONTAL DE CAMBURI            0             0   
        4  2016-04-29T00:00:00Z   56    JARDIM DA PENHA            0             1   
        
           Diabetes  Alcoholism  Handcap  SMS\_received No-show  
        0         0           0        0             0      No  
        1         0           0        0             0      No  
        2         0           0        0             0      No  
        3         0           0        0             0      No  
        4         1           0        0             0      No  
\end{Verbatim}
            
    \begin{Verbatim}[commandchars=\\\{\}]
{\color{incolor}In [{\color{incolor}9}]:} \PY{c+c1}{\PYZsh{} change the data types of ScheduledDay and AppointmentDay to datetime.}
        \PY{k+kn}{from} \PY{n+nn}{datetime} \PY{k}{import} \PY{n}{datetime}
        \PY{n}{df}\PY{p}{[}\PY{l+s+s1}{\PYZsq{}}\PY{l+s+s1}{AppointmentDay}\PY{l+s+s1}{\PYZsq{}}\PY{p}{]} \PY{o}{=} \PY{n}{pd}\PY{o}{.}\PY{n}{to\PYZus{}datetime}\PY{p}{(}\PY{n}{df}\PY{p}{[}\PY{l+s+s1}{\PYZsq{}}\PY{l+s+s1}{AppointmentDay}\PY{l+s+s1}{\PYZsq{}}\PY{p}{]}\PY{p}{)}
        \PY{n}{df}\PY{p}{[}\PY{l+s+s1}{\PYZsq{}}\PY{l+s+s1}{ScheduledDay}\PY{l+s+s1}{\PYZsq{}}\PY{p}{]} \PY{o}{=} \PY{n}{pd}\PY{o}{.}\PY{n}{to\PYZus{}datetime}\PY{p}{(}\PY{n}{df}\PY{p}{[}\PY{l+s+s1}{\PYZsq{}}\PY{l+s+s1}{ScheduledDay}\PY{l+s+s1}{\PYZsq{}}\PY{p}{]}\PY{p}{)}
\end{Verbatim}


    \begin{Verbatim}[commandchars=\\\{\}]
{\color{incolor}In [{\color{incolor}10}]:} \PY{c+c1}{\PYZsh{} Something to note: max age of 115, min age of \PYZhy{}1? See output of df.describe above}
         \PY{c+c1}{\PYZsh{} I wonder how many more centenarians there are and whether this is an outlier...well, }
         \PY{c+c1}{\PYZsh{} the negative person certainly is...unless they considered a fetus. let\PYZsq{}s do some queries}
         \PY{n}{df}\PY{o}{.}\PY{n}{query}\PY{p}{(}\PY{l+s+s1}{\PYZsq{}}\PY{l+s+s1}{Age \PYZgt{}= 100}\PY{l+s+s1}{\PYZsq{}}\PY{p}{)} 
\end{Verbatim}


\begin{Verbatim}[commandchars=\\\{\}]
{\color{outcolor}Out[{\color{outcolor}10}]:}               PatientId  AppointmentID Gender        ScheduledDay  \textbackslash{}
         58014   976294799775439        5651757      F 2016-05-03 09:14:53   
         63912    31963211613981        5700278      F 2016-05-16 09:17:44   
         63915    31963211613981        5700279      F 2016-05-16 09:17:44   
         68127    31963211613981        5562812      F 2016-04-08 14:29:17   
         76284    31963211613981        5744037      F 2016-05-30 09:44:51   
         79270     9739429797896        5747809      M 2016-05-30 16:21:56   
         79272     9739429797896        5747808      M 2016-05-30 16:21:56   
         90372      234283596548        5751563      F 2016-05-31 10:19:49   
         92084    55783129426615        5670914      F 2016-05-06 14:55:36   
         97666   748234579244724        5717451      F 2016-05-19 07:57:56   
         108506     393964189799        5721152      F 2016-05-19 15:32:09   
         
                AppointmentDay  Age    Neighbourhood  Scholarship  Hipertension  \textbackslash{}
         58014      2016-05-03  102        CONQUISTA            0             0   
         63912      2016-05-19  115       ANDORINHAS            0             0   
         63915      2016-05-19  115       ANDORINHAS            0             0   
         68127      2016-05-16  115       ANDORINHAS            0             0   
         76284      2016-05-30  115       ANDORINHAS            0             0   
         79270      2016-05-31  100       TABUAZEIRO            0             0   
         79272      2016-05-31  100       TABUAZEIRO            0             0   
         90372      2016-06-02  102      MARIA ORTIZ            0             0   
         92084      2016-06-03  100  ANTÔNIO HONÓRIO            0             0   
         97666      2016-06-03  115         SÃO JOSÉ            0             1   
         108506     2016-06-01  100          MARUÍPE            0             0   
         
                 Diabetes  Alcoholism  Handcap  SMS\_received No-show  
         58014          0           0        0             0      No  
         63912          0           0        1             0     Yes  
         63915          0           0        1             0     Yes  
         68127          0           0        1             0     Yes  
         76284          0           0        1             0      No  
         79270          0           0        1             0      No  
         79272          0           0        1             0      No  
         90372          0           0        0             0      No  
         92084          0           0        0             1      No  
         97666          0           0        0             1      No  
         108506         0           0        0             0      No  
\end{Verbatim}
            
    There are 7 people over the age of 100, five of whom are 115. These
could be duplicates because the 100 and 102 year olds may indeed be
correct. The author has decided to leave the centenarians as correct but
have a closer look at the 155 yr old. On closer inspection, it looks as
though there are duplicates in the dataset. For example, row 63912 and
63915 look like duplicates created at the point of registration
(probably a call center error?) Two appointments have been created for
the same person: the patientID AND ScheduledDay are the same. The
Appointment ID is different.

    \begin{Verbatim}[commandchars=\\\{\}]
{\color{incolor}In [{\color{incolor}11}]:} \PY{c+c1}{\PYZsh{} How many duplicates in total?}
         \PY{n+nb}{sum}\PY{p}{(}\PY{n}{df}\PY{o}{.}\PY{n}{duplicated}\PY{p}{(}\PY{n}{subset} \PY{o}{=} \PY{p}{[}\PY{l+s+s1}{\PYZsq{}}\PY{l+s+s1}{PatientId}\PY{l+s+s1}{\PYZsq{}}\PY{p}{,} \PY{l+s+s1}{\PYZsq{}}\PY{l+s+s1}{ScheduledDay}\PY{l+s+s1}{\PYZsq{}}\PY{p}{]}\PY{p}{)}\PY{p}{)}
         \PY{c+c1}{\PYZsh{} Have a look at all the duplicates with their originals.}
         \PY{n}{df}\PY{o}{.}\PY{n}{loc}\PY{p}{[}\PY{n}{df}\PY{o}{.}\PY{n}{duplicated}\PY{p}{(}\PY{n}{subset}\PY{o}{=}\PY{p}{[}\PY{l+s+s1}{\PYZsq{}}\PY{l+s+s1}{PatientId}\PY{l+s+s1}{\PYZsq{}}\PY{p}{,} \PY{l+s+s1}{\PYZsq{}}\PY{l+s+s1}{ScheduledDay}\PY{l+s+s1}{\PYZsq{}}\PY{p}{]}\PY{p}{,} \PY{n}{keep} \PY{o}{=} \PY{k+kc}{False}\PY{p}{)}\PY{p}{,} \PY{p}{:}\PY{p}{]}\PY{p}{;}
\end{Verbatim}


    A look through the data suggests that entries can be removed based on
appointmentID, which in each case is different. On the other hand, in
several duplicate observations, the SMS\_recieved is distinct. In some
cases, a text was recieved, in some they weren't. It looks almost as if
the duplication was created as a response to the text message. However,
its not the case for all of them. There are duplications were neither
observation has recorded the SMS being recieved. And a quick glance
shows that not all the SMS being recieved has had an impact on no-show.
In rows 110417 and 110420, although the patient recieved an SMS there
was a no show while she showed up forthe one with the error.

Also, it is not clear to the author why row 367 has been filtered above.

    \begin{Verbatim}[commandchars=\\\{\}]
{\color{incolor}In [{\color{incolor}12}]:} \PY{c+c1}{\PYZsh{} How about the negative age?}
         \PY{n}{df}\PY{o}{.}\PY{n}{query}\PY{p}{(}\PY{l+s+s1}{\PYZsq{}}\PY{l+s+s1}{Age == \PYZhy{}1}\PY{l+s+s1}{\PYZsq{}}\PY{p}{)}
\end{Verbatim}


\begin{Verbatim}[commandchars=\\\{\}]
{\color{outcolor}Out[{\color{outcolor}12}]:}              PatientId  AppointmentID Gender        ScheduledDay  \textbackslash{}
         99832  465943158731293        5775010      F 2016-06-06 08:58:13   
         
               AppointmentDay  Age Neighbourhood  Scholarship  Hipertension  Diabetes  \textbackslash{}
         99832     2016-06-06   -1         ROMÃO            0             0         0   
         
                Alcoholism  Handcap  SMS\_received No-show  
         99832           0        0             0      No  
\end{Verbatim}
            
    The author has decided to leave the negative age in as a possible error
only in age and not in the observation data itself.

    \begin{Verbatim}[commandchars=\\\{\}]
{\color{incolor}In [{\color{incolor}13}]:} \PY{c+c1}{\PYZsh{} Sort the dataset first (so that SMS comes last) then drop the first. This assumes that the first recodring}
         \PY{c+c1}{\PYZsh{} of the patient registration is false. }
         \PY{n}{df}\PY{o}{.}\PY{n}{sort\PYZus{}values}\PY{p}{(}\PY{l+s+s1}{\PYZsq{}}\PY{l+s+s1}{PatientId}\PY{l+s+s1}{\PYZsq{}}\PY{p}{,} \PY{n}{inplace} \PY{o}{=} \PY{k+kc}{True}\PY{p}{)}
         \PY{n}{df}\PY{o}{.}\PY{n}{drop\PYZus{}duplicates}\PY{p}{(}\PY{n}{subset}\PY{o}{=} \PY{p}{[}\PY{l+s+s1}{\PYZsq{}}\PY{l+s+s1}{PatientId}\PY{l+s+s1}{\PYZsq{}}\PY{p}{,} \PY{l+s+s1}{\PYZsq{}}\PY{l+s+s1}{ScheduledDay}\PY{l+s+s1}{\PYZsq{}}\PY{p}{]}\PY{p}{,} \PY{n}{keep}\PY{o}{=}\PY{l+s+s1}{\PYZsq{}}\PY{l+s+s1}{last}\PY{l+s+s1}{\PYZsq{}}\PY{p}{,} \PY{n}{inplace} \PY{o}{=} \PY{k+kc}{True}\PY{p}{)}
         \PY{n}{df}\PY{o}{.}\PY{n}{shape}
\end{Verbatim}


\begin{Verbatim}[commandchars=\\\{\}]
{\color{outcolor}Out[{\color{outcolor}13}]:} (109193, 14)
\end{Verbatim}
            
    \begin{Verbatim}[commandchars=\\\{\}]
{\color{incolor}In [{\color{incolor}14}]:} \PY{c+c1}{\PYZsh{} Store the data so it can be used later without having to go through all the manipulations again}
         \PY{n}{df}\PY{o}{.}\PY{n}{to\PYZus{}csv}\PY{p}{(}\PY{l+s+s1}{\PYZsq{}}\PY{l+s+s1}{data/noShow\PYZus{}clean.csv}\PY{l+s+s1}{\PYZsq{}}\PY{p}{,} \PY{n}{index}\PY{o}{=}\PY{k+kc}{False}\PY{p}{)}
\end{Verbatim}


     \#\# Exploratory Data Analysis

The main question to be answered in the no-show data set is: "What
factors are important for us to know in order to predict if a patient
will show up for their scheduled appointment?" As a predictive model
will not be created here at this time, the research question is divided
into the preliminary analysis required before creating a predictive
model.

\subsubsection{Research question 1: What factors are highly correlated
to
outcome?}\label{research-question-1-what-factors-are-highly-correlated-to-outcome}

    \begin{Verbatim}[commandchars=\\\{\}]
{\color{incolor}In [{\color{incolor}15}]:} \PY{c+c1}{\PYZsh{} read in the sorted data and make sure its all ok}
         \PY{n}{nosho} \PY{o}{=} \PY{n}{pd}\PY{o}{.}\PY{n}{read\PYZus{}csv}\PY{p}{(}\PY{l+s+s2}{\PYZdq{}}\PY{l+s+s2}{data/noShow\PYZus{}clean.csv}\PY{l+s+s2}{\PYZdq{}}\PY{p}{)}
         \PY{c+c1}{\PYZsh{} check to see it\PYZsq{}s all ok}
         \PY{n}{nosho}\PY{o}{.}\PY{n}{shape}
         \PY{n}{nosho}\PY{o}{.}\PY{n}{head}\PY{p}{(}\PY{p}{)}
         \PY{n}{nosho}\PY{o}{.}\PY{n}{dtypes}
\end{Verbatim}


\begin{Verbatim}[commandchars=\\\{\}]
{\color{outcolor}Out[{\color{outcolor}15}]:} (109193, 14)
\end{Verbatim}
            
\begin{Verbatim}[commandchars=\\\{\}]
{\color{outcolor}Out[{\color{outcolor}15}]:}    PatientId  AppointmentID Gender         ScheduledDay       AppointmentDay  \textbackslash{}
         0      39217        5751990      F  2016-05-31 10:56:41  2016-06-03 00:00:00   
         1      43741        5760144      M  2016-06-01 14:22:58  2016-06-01 00:00:00   
         2      93779        5712759      F  2016-05-18 09:12:29  2016-05-18 00:00:00   
         3     141724        5637648      M  2016-04-29 07:13:36  2016-05-02 00:00:00   
         4     537615        5637728      F  2016-04-29 07:19:57  2016-05-06 00:00:00   
         
            Age   Neighbourhood  Scholarship  Hipertension  Diabetes  Alcoholism  \textbackslash{}
         0   44    PRAIA DO SUÁ            0             0         0           0   
         1   39     MARIA ORTIZ            0             0         1           0   
         2   33          CENTRO            0             0         0           0   
         3   12  FORTE SÃO JOÃO            0             0         0           0   
         4   14  FORTE SÃO JOÃO            0             0         0           0   
         
            Handcap  SMS\_received No-show  
         0        0             0      No  
         1        0             0      No  
         2        0             0      No  
         3        0             0      No  
         4        0             1      No  
\end{Verbatim}
            
\begin{Verbatim}[commandchars=\\\{\}]
{\color{outcolor}Out[{\color{outcolor}15}]:} PatientId          int64
         AppointmentID      int64
         Gender            object
         ScheduledDay      object
         AppointmentDay    object
         Age                int64
         Neighbourhood     object
         Scholarship        int64
         Hipertension       int64
         Diabetes           int64
         Alcoholism         int64
         Handcap            int64
         SMS\_received       int64
         No-show           object
         dtype: object
\end{Verbatim}
            
    \begin{Verbatim}[commandchars=\\\{\}]
{\color{incolor}In [{\color{incolor}16}]:} \PY{c+c1}{\PYZsh{} looks like the data types from before weren\PYZsq{}t preserved. }
         \PY{n}{nosho}\PY{p}{[}\PY{l+s+s1}{\PYZsq{}}\PY{l+s+s1}{AppointmentDay}\PY{l+s+s1}{\PYZsq{}}\PY{p}{]} \PY{o}{=} \PY{n}{pd}\PY{o}{.}\PY{n}{to\PYZus{}datetime}\PY{p}{(}\PY{n}{nosho}\PY{p}{[}\PY{l+s+s1}{\PYZsq{}}\PY{l+s+s1}{AppointmentDay}\PY{l+s+s1}{\PYZsq{}}\PY{p}{]}\PY{p}{)}
         \PY{n}{nosho}\PY{p}{[}\PY{l+s+s1}{\PYZsq{}}\PY{l+s+s1}{ScheduledDay}\PY{l+s+s1}{\PYZsq{}}\PY{p}{]} \PY{o}{=} \PY{n}{pd}\PY{o}{.}\PY{n}{to\PYZus{}datetime}\PY{p}{(}\PY{n}{nosho}\PY{p}{[}\PY{l+s+s1}{\PYZsq{}}\PY{l+s+s1}{ScheduledDay}\PY{l+s+s1}{\PYZsq{}}\PY{p}{]}\PY{p}{)}
\end{Verbatim}


    \begin{Verbatim}[commandchars=\\\{\}]
{\color{incolor}In [{\color{incolor}44}]:} \PY{c+c1}{\PYZsh{} To do any correlations the categorical variables must first be changed to dummies. }
         \PY{c+c1}{\PYZsh{} This is not always recommended when attempting predictive analysis but as this is just to look at correlations,}
         \PY{c+c1}{\PYZsh{} it shouldn\PYZsq{}t matter too much}
         \PY{n}{nosho} \PY{o}{=} \PY{n}{nosho}\PY{o}{.}\PY{n}{join}\PY{p}{(}\PY{n}{pd}\PY{o}{.}\PY{n}{get\PYZus{}dummies}\PY{p}{(}\PY{n}{nosho}\PY{p}{[}\PY{l+s+s1}{\PYZsq{}}\PY{l+s+s1}{Gender}\PY{l+s+s1}{\PYZsq{}}\PY{p}{]}\PY{p}{)}\PY{p}{)}
         \PY{n}{nosho} \PY{o}{=} \PY{n}{nosho}\PY{o}{.}\PY{n}{join}\PY{p}{(}\PY{n}{pd}\PY{o}{.}\PY{n}{get\PYZus{}dummies}\PY{p}{(}\PY{n}{nosho}\PY{p}{[}\PY{l+s+s1}{\PYZsq{}}\PY{l+s+s1}{No\PYZhy{}show}\PY{l+s+s1}{\PYZsq{}}\PY{p}{]}\PY{p}{)}\PY{p}{)}
         \PY{n}{nosho}\PY{o}{.}\PY{n}{head}\PY{p}{(}\PY{l+m+mi}{10}\PY{p}{)}
         \PY{c+c1}{\PYZsh{} note that now the dataset contains four extra columns. No\PYZhy{}show has been split into Yes (failed to attend) }
         \PY{c+c1}{\PYZsh{} and No (attended), and Gender into M and F. }
\end{Verbatim}


\begin{Verbatim}[commandchars=\\\{\}]
{\color{outcolor}Out[{\color{outcolor}44}]:}    PatientId  AppointmentID Gender        ScheduledDay AppointmentDay  Age  \textbackslash{}
         0      39217        5751990      F 2016-05-31 10:56:41     2016-06-03   44   
         1      43741        5760144      M 2016-06-01 14:22:58     2016-06-01   39   
         2      93779        5712759      F 2016-05-18 09:12:29     2016-05-18   33   
         3     141724        5637648      M 2016-04-29 07:13:36     2016-05-02   12   
         4     537615        5637728      F 2016-04-29 07:19:57     2016-05-06   14   
         5    5628261        5680449      M 2016-05-10 11:58:18     2016-05-13   13   
         6   11831856        5718578      M 2016-05-19 09:42:07     2016-05-19   16   
         7   22638656        5715081      F 2016-05-18 13:37:12     2016-06-08   23   
         8   22638656        5580835      F 2016-04-14 07:23:30     2016-05-03   22   
         9   52168938        5607220      F 2016-04-20 11:22:15     2016-05-17   28   
         
              Neighbourhood  Scholarship  Hipertension  Diabetes  Alcoholism  Handcap  \textbackslash{}
         0     PRAIA DO SUÁ            0             0         0           0        0   
         1      MARIA ORTIZ            0             0         1           0        0   
         2           CENTRO            0             0         0           0        0   
         3   FORTE SÃO JOÃO            0             0         0           0        0   
         4   FORTE SÃO JOÃO            0             0         0           0        0   
         5   PARQUE MOSCOSO            0             0         0           0        0   
         6    SANTO ANTÔNIO            0             0         0           0        0   
         7       INHANGUETÁ            0             0         0           0        0   
         8       INHANGUETÁ            0             0         0           0        0   
         9  JARDIM DA PENHA            0             0         0           0        0   
         
            SMS\_received No-show  F  M  No  Yes  
         0             0      No  1  0   1    0  
         1             0      No  0  1   1    0  
         2             0      No  1  0   1    0  
         3             0      No  0  1   1    0  
         4             1      No  1  0   1    0  
         5             0     Yes  0  1   0    1  
         6             0      No  0  1   1    0  
         7             1      No  1  0   1    0  
         8             1      No  1  0   1    0  
         9             0      No  1  0   1    0  
\end{Verbatim}
            
    Before doing any correlations, it is important to see how the outcome
itself is distributed. In particular, bias needs to be ascertained in
the dataset.

    \begin{Verbatim}[commandchars=\\\{\}]
{\color{incolor}In [{\color{incolor}72}]:} \PY{c+c1}{\PYZsh{} Let\PYZsq{}s have a look at distribution frequencies using crosstab to tally counts}
         \PY{c+c1}{\PYZsh{} first how many no\PYZhy{}shows and show\PYZsq{}s are there?}
         \PY{n}{fig} \PY{o}{=} \PY{n}{plt}\PY{o}{.}\PY{n}{figure}\PY{p}{(}\PY{p}{)}\PY{p}{;}
         \PY{n}{nosho\PYZus{}table} \PY{o}{=} \PY{n}{pd}\PY{o}{.}\PY{n}{crosstab}\PY{p}{(}\PY{n}{index}\PY{o}{=}\PY{n}{nosho}\PY{p}{[}\PY{l+s+s2}{\PYZdq{}}\PY{l+s+s2}{No\PYZhy{}show}\PY{l+s+s2}{\PYZdq{}}\PY{p}{]}\PY{p}{,} \PY{n}{columns}\PY{o}{=}\PY{l+s+s2}{\PYZdq{}}\PY{l+s+s2}{count}\PY{l+s+s2}{\PYZdq{}}\PY{p}{)}
         \PY{n}{nosho\PYZus{}table}
         \PY{n}{nosho\PYZus{}table}\PY{o}{.}\PY{n}{plot}\PY{p}{(}\PY{n}{kind}\PY{o}{=}\PY{l+s+s2}{\PYZdq{}}\PY{l+s+s2}{bar}\PY{l+s+s2}{\PYZdq{}}\PY{p}{,}\PY{n}{figsize}\PY{o}{=}\PY{p}{(}\PY{l+m+mi}{8}\PY{p}{,}\PY{l+m+mi}{8}\PY{p}{)}\PY{p}{)}\PY{p}{;}
         \PY{n}{plt}\PY{o}{.}\PY{n}{title}\PY{p}{(}\PY{l+s+s1}{\PYZsq{}}\PY{l+s+s1}{Figure 1: Number of attendance and cancellations}\PY{l+s+s1}{\PYZsq{}}\PY{p}{,} \PY{n}{fontsize} \PY{o}{=} \PY{l+m+mi}{16}\PY{p}{)}\PY{p}{;}
         \PY{n}{txt} \PY{o}{=} \PY{l+s+s2}{\PYZdq{}}\PY{l+s+s2}{Figure 1: This figure shows the total numbers of attendance and cancellations across the dataset}\PY{l+s+s2}{\PYZdq{}}\PY{p}{;}
         \PY{n}{fig}\PY{o}{.}\PY{n}{text}\PY{p}{(}\PY{o}{.}\PY{l+m+mi}{005}\PY{p}{,} \PY{o}{.}\PY{l+m+mi}{01}\PY{p}{,} \PY{n}{txt}\PY{p}{,} \PY{n}{ha}\PY{o}{=}\PY{l+s+s1}{\PYZsq{}}\PY{l+s+s1}{center}\PY{l+s+s1}{\PYZsq{}}\PY{p}{,} \PY{n}{fontsize} \PY{o}{=} \PY{l+m+mi}{14}\PY{p}{,} \PY{n}{wrap} \PY{o}{=} \PY{k+kc}{True}\PY{p}{)}\PY{p}{;}
\end{Verbatim}


    
    \begin{verbatim}
<matplotlib.figure.Figure at 0x10dbf5dd8>
    \end{verbatim}

    
    \begin{center}
    \adjustimage{max size={0.9\linewidth}{0.9\paperheight}}{output_28_1.png}
    \end{center}
    { \hspace*{\fill} \\}
    
    Figure 1 shows the total numbers of attendance and cancellations across
the dataset. From this it can be noted that there will be a bias in this
dataset and that it will impact any correlations and regressions
attempted with the data.

    \begin{Verbatim}[commandchars=\\\{\}]
{\color{incolor}In [{\color{incolor}73}]:} \PY{c+c1}{\PYZsh{} Let\PYZsq{}s have a look at some quick visual histograms. }
         \PY{c+c1}{\PYZsh{} Most of the variables here are categorical but a histogram of age can be plotted}
         \PY{n}{fig} \PY{o}{=} \PY{n}{plt}\PY{o}{.}\PY{n}{figure}\PY{p}{(}\PY{p}{)}\PY{p}{;}
         \PY{n}{nosho}\PY{p}{[}\PY{l+s+s1}{\PYZsq{}}\PY{l+s+s1}{Age}\PY{l+s+s1}{\PYZsq{}}\PY{p}{]}\PY{o}{.}\PY{n}{hist}\PY{p}{(}\PY{p}{)}\PY{p}{;}
         \PY{n}{plt}\PY{o}{.}\PY{n}{ylabel}\PY{p}{(}\PY{l+s+s1}{\PYZsq{}}\PY{l+s+s1}{Observations}\PY{l+s+s1}{\PYZsq{}}\PY{p}{)}\PY{p}{;}
         \PY{n}{plt}\PY{o}{.}\PY{n}{xlabel}\PY{p}{(}\PY{l+s+s1}{\PYZsq{}}\PY{l+s+s1}{Age}\PY{l+s+s1}{\PYZsq{}}\PY{p}{)}\PY{p}{;}
         \PY{n}{plt}\PY{o}{.}\PY{n}{title}\PY{p}{(}\PY{l+s+s1}{\PYZsq{}}\PY{l+s+s1}{Figure 2: Age distribution}\PY{l+s+s1}{\PYZsq{}}\PY{p}{,} \PY{n}{fontsize} \PY{o}{=} \PY{l+m+mi}{16}\PY{p}{)}\PY{p}{;}
         \PY{n}{txt} \PY{o}{=} \PY{l+s+s2}{\PYZdq{}}\PY{l+s+s2}{Figure 2: This figure shows age distribution and it looks right skewed.}\PY{l+s+s2}{\PYZdq{}}\PY{p}{;}
         \PY{n}{fig}\PY{o}{.}\PY{n}{text}\PY{p}{(}\PY{o}{.}\PY{l+m+mi}{5}\PY{p}{,} \PY{o}{.}\PY{l+m+mi}{001}\PY{p}{,} \PY{n}{txt}\PY{p}{,} \PY{n}{ha}\PY{o}{=}\PY{l+s+s1}{\PYZsq{}}\PY{l+s+s1}{center}\PY{l+s+s1}{\PYZsq{}}\PY{p}{,} \PY{n}{fontsize} \PY{o}{=} \PY{l+m+mi}{14}\PY{p}{,} \PY{n}{wrap} \PY{o}{=} \PY{k+kc}{True}\PY{p}{)}\PY{p}{;}
\end{Verbatim}


    \begin{center}
    \adjustimage{max size={0.9\linewidth}{0.9\paperheight}}{output_30_0.png}
    \end{center}
    { \hspace*{\fill} \\}
    
    The Age distribution shown in the figure above is right skewed
(positively skewed) and the mean is probably to the right of the median.
This means that the sample sizes are larger for ages 0-18 and 35-50,
than for the older ages. This needs to be taken into account before
drawing conclusions on correlation

    \begin{Verbatim}[commandchars=\\\{\}]
{\color{incolor}In [{\color{incolor}76}]:} \PY{c+c1}{\PYZsh{} verifing distribution on a boxplot: positive skewness}
         \PY{n}{fig} \PY{o}{=} \PY{n}{plt}\PY{o}{.}\PY{n}{figure}\PY{p}{(}\PY{p}{)}\PY{p}{;}
         \PY{n}{nosho}\PY{o}{.}\PY{n}{boxplot}\PY{p}{(}\PY{l+s+s1}{\PYZsq{}}\PY{l+s+s1}{Age}\PY{l+s+s1}{\PYZsq{}}\PY{p}{,} \PY{n}{vert} \PY{o}{=} \PY{k+kc}{False}\PY{p}{)}\PY{p}{;}
         \PY{n}{plt}\PY{o}{.}\PY{n}{title}\PY{p}{(}\PY{l+s+s1}{\PYZsq{}}\PY{l+s+s1}{Figure 3: Age distribution showing skewness}\PY{l+s+s1}{\PYZsq{}}\PY{p}{,} \PY{n}{fontsize} \PY{o}{=} \PY{l+m+mi}{16}\PY{p}{)}\PY{p}{;}
         \PY{n}{plt}\PY{o}{.}\PY{n}{xlabel}\PY{p}{(}\PY{l+s+s1}{\PYZsq{}}\PY{l+s+s1}{Observations}\PY{l+s+s1}{\PYZsq{}}\PY{p}{)}
         \PY{n}{txt} \PY{o}{=} \PY{l+s+s2}{\PYZdq{}}\PY{l+s+s2}{Figure 3: This is a horizontal boxplot of age distribution and it complements figure 1 to show a right skewed distribution.}\PY{l+s+s2}{\PYZdq{}}\PY{p}{;}
         \PY{n}{fig}\PY{o}{.}\PY{n}{text}\PY{p}{(}\PY{o}{.}\PY{l+m+mi}{5}\PY{p}{,} \PY{o}{.}\PY{l+m+mi}{001}\PY{p}{,} \PY{n}{txt}\PY{p}{,} \PY{n}{ha}\PY{o}{=}\PY{l+s+s1}{\PYZsq{}}\PY{l+s+s1}{center}\PY{l+s+s1}{\PYZsq{}}\PY{p}{,} \PY{n}{fontsize} \PY{o}{=} \PY{l+m+mi}{14}\PY{p}{,} \PY{n}{wrap} \PY{o}{=} \PY{k+kc}{True}\PY{p}{)}\PY{p}{;}
\end{Verbatim}


    \begin{center}
    \adjustimage{max size={0.9\linewidth}{0.9\paperheight}}{output_32_0.png}
    \end{center}
    { \hspace*{\fill} \\}
    
    Figure 3 complements figure 2. It shows the right skewed or positive
skewness of the age, where the mean is very likely to the right of the
median. This means that the sample sizes are much larger in the younger
ages rather than in the older ones and needs ot be taken into account
before drawing conculsions on correlations

    \begin{Verbatim}[commandchars=\\\{\}]
{\color{incolor}In [{\color{incolor}59}]:} \PY{c+c1}{\PYZsh{}correlation between age and attendance}
         \PY{n}{nosho}\PY{p}{[}\PY{p}{[}\PY{l+s+s1}{\PYZsq{}}\PY{l+s+s1}{Age}\PY{l+s+s1}{\PYZsq{}}\PY{p}{,}\PY{l+s+s1}{\PYZsq{}}\PY{l+s+s1}{Yes}\PY{l+s+s1}{\PYZsq{}}\PY{p}{,} \PY{l+s+s1}{\PYZsq{}}\PY{l+s+s1}{No}\PY{l+s+s1}{\PYZsq{}}\PY{p}{]}\PY{p}{]}\PY{o}{.}\PY{n}{corr}\PY{p}{(}\PY{p}{)}
\end{Verbatim}


\begin{Verbatim}[commandchars=\\\{\}]
{\color{outcolor}Out[{\color{outcolor}59}]:}           Age       Yes        No
         Age  1.000000 -0.060768  0.060768
         Yes -0.060768  1.000000 -1.000000
         No   0.060768 -1.000000  1.000000
\end{Verbatim}
            
    The table above shows that there is little correlation between age,
attendance and non-attendance. The p-coefficient when using Pearson
Correlation should be around 0.7 or higher to be considered significant*
(See note in conclusion on using this statistical method with
categorical date)

    \begin{Verbatim}[commandchars=\\\{\}]
{\color{incolor}In [{\color{incolor}48}]:} \PY{c+c1}{\PYZsh{} It may be more interesting to cut the ages and see if there is any correlation}
         \PY{n}{bins} \PY{o}{=} \PY{p}{[}\PY{l+m+mi}{0}\PY{p}{,} \PY{l+m+mi}{18}\PY{p}{,} \PY{l+m+mi}{30}\PY{p}{,} \PY{l+m+mi}{45}\PY{p}{,} \PY{l+m+mi}{55}\PY{p}{,} \PY{l+m+mi}{65}\PY{p}{,} \PY{l+m+mi}{75}\PY{p}{,} \PY{l+m+mi}{100}\PY{p}{]}
         \PY{n}{age\PYZus{}groups} \PY{o}{=} \PY{p}{[}\PY{l+s+s1}{\PYZsq{}}\PY{l+s+s1}{0\PYZhy{}18}\PY{l+s+s1}{\PYZsq{}}\PY{p}{,} \PY{l+s+s1}{\PYZsq{}}\PY{l+s+s1}{18\PYZhy{}30}\PY{l+s+s1}{\PYZsq{}}\PY{p}{,} \PY{l+s+s1}{\PYZsq{}}\PY{l+s+s1}{30\PYZhy{}45}\PY{l+s+s1}{\PYZsq{}}\PY{p}{,} \PY{l+s+s1}{\PYZsq{}}\PY{l+s+s1}{45\PYZhy{}55}\PY{l+s+s1}{\PYZsq{}}\PY{p}{,} \PY{l+s+s1}{\PYZsq{}}\PY{l+s+s1}{55\PYZhy{}65}\PY{l+s+s1}{\PYZsq{}}\PY{p}{,} \PY{l+s+s1}{\PYZsq{}}\PY{l+s+s1}{65\PYZhy{}75}\PY{l+s+s1}{\PYZsq{}}\PY{p}{,} \PY{l+s+s1}{\PYZsq{}}\PY{l+s+s1}{75+}\PY{l+s+s1}{\PYZsq{}}\PY{p}{]}
         \PY{n}{nosho}\PY{p}{[}\PY{l+s+s1}{\PYZsq{}}\PY{l+s+s1}{age\PYZus{}groups}\PY{l+s+s1}{\PYZsq{}}\PY{p}{]} \PY{o}{=} \PY{n}{pd}\PY{o}{.}\PY{n}{cut}\PY{p}{(}\PY{n}{nosho}\PY{p}{[}\PY{l+s+s1}{\PYZsq{}}\PY{l+s+s1}{Age}\PY{l+s+s1}{\PYZsq{}}\PY{p}{]}\PY{p}{,} \PY{n}{bins}\PY{p}{,} \PY{n}{labels}\PY{o}{=}\PY{n}{age\PYZus{}groups}\PY{p}{)}
\end{Verbatim}


    \begin{Verbatim}[commandchars=\\\{\}]
{\color{incolor}In [{\color{incolor}80}]:} \PY{c+c1}{\PYZsh{} let\PYZsq{}s have a quick look at counts and head, to check that everything binned ok}
         \PY{n}{pd}\PY{o}{.}\PY{n}{value\PYZus{}counts}\PY{p}{(}\PY{n}{nosho}\PY{p}{[}\PY{l+s+s1}{\PYZsq{}}\PY{l+s+s1}{age\PYZus{}groups}\PY{l+s+s1}{\PYZsq{}}\PY{p}{]}\PY{p}{)}
         \PY{n}{nosho}\PY{o}{.}\PY{n}{head}\PY{p}{(}\PY{p}{)}
\end{Verbatim}


\begin{Verbatim}[commandchars=\\\{\}]
{\color{outcolor}Out[{\color{outcolor}80}]:} 0-18     25040
         30-45    21601
         18-30    16536
         45-55    15237
         55-65    14050
         65-75     7851
         75+       5347
         Name: age\_groups, dtype: int64
\end{Verbatim}
            
\begin{Verbatim}[commandchars=\\\{\}]
{\color{outcolor}Out[{\color{outcolor}80}]:}    PatientId  AppointmentID Gender        ScheduledDay AppointmentDay  Age  \textbackslash{}
         0      39217        5751990      F 2016-05-31 10:56:41     2016-06-03   44   
         1      43741        5760144      M 2016-06-01 14:22:58     2016-06-01   39   
         2      93779        5712759      F 2016-05-18 09:12:29     2016-05-18   33   
         3     141724        5637648      M 2016-04-29 07:13:36     2016-05-02   12   
         4     537615        5637728      F 2016-04-29 07:19:57     2016-05-06   14   
         
             Neighbourhood  Scholarship  Hipertension  Diabetes  Alcoholism  Handcap  \textbackslash{}
         0    PRAIA DO SUÁ            0             0         0           0        0   
         1     MARIA ORTIZ            0             0         1           0        0   
         2          CENTRO            0             0         0           0        0   
         3  FORTE SÃO JOÃO            0             0         0           0        0   
         4  FORTE SÃO JOÃO            0             0         0           0        0   
         
            SMS\_received No-show  F  M  No  Yes age\_groups  
         0             0      No  1  0   1    0      30-45  
         1             0      No  0  1   1    0      30-45  
         2             0      No  1  0   1    0      30-45  
         3             0      No  0  1   1    0       0-18  
         4             1      No  1  0   1    0       0-18  
\end{Verbatim}
            
    \begin{Verbatim}[commandchars=\\\{\}]
{\color{incolor}In [{\color{incolor}75}]:} \PY{c+c1}{\PYZsh{} now let\PYZsq{}s group the data within ages and see if any conclusions can be drawn}
         \PY{n}{fig} \PY{o}{=} \PY{n}{plt}\PY{o}{.}\PY{n}{figure}\PY{p}{(}\PY{p}{)}\PY{p}{;}
         \PY{n}{nosho\PYZus{}ages} \PY{o}{=} \PY{n}{nosho}\PY{o}{.}\PY{n}{groupby}\PY{p}{(}\PY{l+s+s1}{\PYZsq{}}\PY{l+s+s1}{age\PYZus{}groups}\PY{l+s+s1}{\PYZsq{}}\PY{p}{)}\PY{p}{[}\PY{l+s+s1}{\PYZsq{}}\PY{l+s+s1}{No}\PY{l+s+s1}{\PYZsq{}}\PY{p}{,}\PY{l+s+s1}{\PYZsq{}}\PY{l+s+s1}{Yes}\PY{l+s+s1}{\PYZsq{}}\PY{p}{]}\PY{o}{.}\PY{n}{sum}\PY{p}{(}\PY{p}{)}\PY{p}{;}
         \PY{n}{nosho\PYZus{}ages}\PY{p}{[}\PY{p}{[}\PY{l+s+s1}{\PYZsq{}}\PY{l+s+s1}{No}\PY{l+s+s1}{\PYZsq{}}\PY{p}{,}\PY{l+s+s1}{\PYZsq{}}\PY{l+s+s1}{Yes}\PY{l+s+s1}{\PYZsq{}}\PY{p}{]}\PY{p}{]}\PY{o}{.}\PY{n}{plot}\PY{p}{(}\PY{n}{kind}\PY{o}{=}\PY{l+s+s1}{\PYZsq{}}\PY{l+s+s1}{bar}\PY{l+s+s1}{\PYZsq{}}\PY{p}{,}\PY{n}{stacked}\PY{o}{=}\PY{k+kc}{True}\PY{p}{)}\PY{p}{;}
         \PY{n}{plt}\PY{o}{.}\PY{n}{ylabel}\PY{p}{(}\PY{l+s+s1}{\PYZsq{}}\PY{l+s+s1}{Observations}\PY{l+s+s1}{\PYZsq{}}\PY{p}{)}\PY{p}{;}
         \PY{n}{plt}\PY{o}{.}\PY{n}{title}\PY{p}{(}\PY{l+s+s1}{\PYZsq{}}\PY{l+s+s1}{Figure 4: Observations by age groups}\PY{l+s+s1}{\PYZsq{}}\PY{p}{,} \PY{n}{fontsize} \PY{o}{=} \PY{l+m+mi}{18}\PY{p}{)}\PY{p}{;}
         \PY{n}{txt} \PY{o}{=} \PY{l+s+s2}{\PYZdq{}}\PY{l+s+s2}{Figure 4: This is a stacked bar graph showing distribution of age groups.}\PY{l+s+s2}{\PYZdq{}}\PY{p}{;}
         \PY{n}{fig}\PY{o}{.}\PY{n}{text}\PY{p}{(}\PY{o}{.}\PY{l+m+mi}{1}\PY{p}{,} \PY{o}{.}\PY{l+m+mi}{1}\PY{p}{,} \PY{n}{txt}\PY{p}{,} \PY{n}{ha}\PY{o}{=}\PY{l+s+s1}{\PYZsq{}}\PY{l+s+s1}{center}\PY{l+s+s1}{\PYZsq{}}\PY{p}{,} \PY{n}{fontsize} \PY{o}{=} \PY{l+m+mi}{14}\PY{p}{,} \PY{n}{wrap} \PY{o}{=} \PY{k+kc}{True}\PY{p}{)}\PY{p}{;}
         \PY{n}{nosho\PYZus{}ages}
\end{Verbatim}


\begin{Verbatim}[commandchars=\\\{\}]
{\color{outcolor}Out[{\color{outcolor}75}]:} <matplotlib.axes.\_subplots.AxesSubplot at 0x10f169898>
\end{Verbatim}
            
\begin{Verbatim}[commandchars=\\\{\}]
{\color{outcolor}Out[{\color{outcolor}75}]:} <matplotlib.text.Text at 0x10db3a320>
\end{Verbatim}
            
\begin{Verbatim}[commandchars=\\\{\}]
{\color{outcolor}Out[{\color{outcolor}75}]:} <matplotlib.text.Text at 0x10e9afba8>
\end{Verbatim}
            
\begin{Verbatim}[commandchars=\\\{\}]
{\color{outcolor}Out[{\color{outcolor}75}]:} <matplotlib.text.Text at 0x10e64a400>
\end{Verbatim}
            
\begin{Verbatim}[commandchars=\\\{\}]
{\color{outcolor}Out[{\color{outcolor}75}]:}                  No     Yes
         age\_groups                 
         0-18        19433.0  5607.0
         18-30       12473.0  4063.0
         30-45       17036.0  4565.0
         45-55       12463.0  2774.0
         55-65       11860.0  2190.0
         65-75        6670.0  1181.0
         75+          4497.0   850.0
\end{Verbatim}
            
    
    \begin{verbatim}
<matplotlib.figure.Figure at 0x10db43a58>
    \end{verbatim}

    
    \begin{center}
    \adjustimage{max size={0.9\linewidth}{0.9\paperheight}}{output_38_6.png}
    \end{center}
    { \hspace*{\fill} \\}
    
    Figure 4 is a stacked bar graph that shows age distribution according to
groups. It also shows the attendance and no-attendance figures. It could
be concluded from the graph that the younger age groups are more likely
to cancel attendanc ebut it must be noted that the sample size is much
larger in these groups and distribution skewed as was seen in figures 1
and 2. The table above the figure displays the counts in each age group
and supports the remark on sample sizes

    It will be interesting to see if gender has an impact in attendance.

    \begin{Verbatim}[commandchars=\\\{\}]
{\color{incolor}In [{\color{incolor}79}]:} \PY{c+c1}{\PYZsh{} Find the total counts of male and female}
         \PY{n}{fig} \PY{o}{=} \PY{n}{plt}\PY{o}{.}\PY{n}{figure}\PY{p}{(}\PY{p}{)}\PY{p}{;}
         \PY{n}{nosho\PYZus{}mf} \PY{o}{=} \PY{n}{pd}\PY{o}{.}\PY{n}{crosstab}\PY{p}{(}\PY{n}{index}\PY{o}{=}\PY{n}{nosho}\PY{p}{[}\PY{l+s+s2}{\PYZdq{}}\PY{l+s+s2}{Gender}\PY{l+s+s2}{\PYZdq{}}\PY{p}{]}\PY{p}{,} \PY{n}{columns}\PY{o}{=}\PY{l+s+s2}{\PYZdq{}}\PY{l+s+s2}{count}\PY{l+s+s2}{\PYZdq{}}\PY{p}{)}
         \PY{n}{nosho\PYZus{}mf}\PY{o}{.}\PY{n}{plot}\PY{p}{(}\PY{n}{kind}\PY{o}{=}\PY{l+s+s2}{\PYZdq{}}\PY{l+s+s2}{bar}\PY{l+s+s2}{\PYZdq{}}\PY{p}{,}\PY{n}{figsize}\PY{o}{=}\PY{p}{(}\PY{l+m+mi}{8}\PY{p}{,}\PY{l+m+mi}{8}\PY{p}{)}\PY{p}{)}\PY{p}{;}
         \PY{n}{plt}\PY{o}{.}\PY{n}{title}\PY{p}{(}\PY{l+s+s1}{\PYZsq{}}\PY{l+s+s1}{Figure 5: Gender mix}\PY{l+s+s1}{\PYZsq{}}\PY{p}{,} \PY{n}{fontsize} \PY{o}{=} \PY{l+m+mi}{16}\PY{p}{)}\PY{p}{;}
         \PY{n}{txt} \PY{o}{=} \PY{l+s+s2}{\PYZdq{}}\PY{l+s+s2}{Figure 5: This is a bar graph showing total gender mix}\PY{l+s+s2}{\PYZdq{}}\PY{p}{;}
         \PY{n}{fig}\PY{o}{.}\PY{n}{text}\PY{p}{(}\PY{o}{.}\PY{l+m+mi}{1}\PY{p}{,} \PY{o}{.}\PY{l+m+mi}{1}\PY{p}{,} \PY{n}{txt}\PY{p}{,} \PY{n}{ha}\PY{o}{=}\PY{l+s+s1}{\PYZsq{}}\PY{l+s+s1}{center}\PY{l+s+s1}{\PYZsq{}}\PY{p}{,} \PY{n}{fontsize} \PY{o}{=} \PY{l+m+mi}{14}\PY{p}{,} \PY{n}{wrap} \PY{o}{=} \PY{k+kc}{True}\PY{p}{)}\PY{p}{;}
         \PY{n}{plt}\PY{o}{.}\PY{n}{ylabel}\PY{p}{(}\PY{l+s+s1}{\PYZsq{}}\PY{l+s+s1}{Observations}\PY{l+s+s1}{\PYZsq{}}\PY{p}{)}\PY{p}{;}
\end{Verbatim}


    
    \begin{verbatim}
<matplotlib.figure.Figure at 0x10e1705f8>
    \end{verbatim}

    
    \begin{center}
    \adjustimage{max size={0.9\linewidth}{0.9\paperheight}}{output_41_1.png}
    \end{center}
    { \hspace*{\fill} \\}
    
    Figure 5 is a bar graph showing total gender mix. Again, it is important
to see here that the sample size for females is greater than males and
this needs to be taken into account when drawing any conclusions on
corrleations

    \begin{Verbatim}[commandchars=\\\{\}]
{\color{incolor}In [{\color{incolor}60}]:} \PY{c+c1}{\PYZsh{} gender, age and attendance}
         \PY{n}{nosho}\PY{p}{[}\PY{p}{[}\PY{l+s+s1}{\PYZsq{}}\PY{l+s+s1}{Age}\PY{l+s+s1}{\PYZsq{}}\PY{p}{,}\PY{l+s+s1}{\PYZsq{}}\PY{l+s+s1}{F}\PY{l+s+s1}{\PYZsq{}}\PY{p}{,}\PY{l+s+s1}{\PYZsq{}}\PY{l+s+s1}{M}\PY{l+s+s1}{\PYZsq{}}\PY{p}{,}\PY{l+s+s1}{\PYZsq{}}\PY{l+s+s1}{Yes}\PY{l+s+s1}{\PYZsq{}}\PY{p}{]}\PY{p}{]}\PY{o}{.}\PY{n}{corr}\PY{p}{(}\PY{p}{)}
\end{Verbatim}


\begin{Verbatim}[commandchars=\\\{\}]
{\color{outcolor}Out[{\color{outcolor}60}]:}           Age         F         M       Yes
         Age  1.000000  0.107164 -0.107164 -0.060768
         F    0.107164  1.000000 -1.000000  0.004807
         M   -0.107164 -1.000000  1.000000 -0.004807
         Yes -0.060768  0.004807 -0.004807  1.000000
\end{Verbatim}
            
    In the table above, it can be concluded that there is no correlation
between gender and non-attendance. On the other hand, it must be noted
that there is a larger sample size of females in the dataset, so any
kind of conclusion drawn comes with a bias

    \begin{Verbatim}[commandchars=\\\{\}]
{\color{incolor}In [{\color{incolor}65}]:} \PY{c+c1}{\PYZsh{} correlation between SMS reminders and attendance}
         \PY{n}{nosho}\PY{p}{[}\PY{p}{[}\PY{l+s+s1}{\PYZsq{}}\PY{l+s+s1}{SMS\PYZus{}received}\PY{l+s+s1}{\PYZsq{}}\PY{p}{,}\PY{l+s+s1}{\PYZsq{}}\PY{l+s+s1}{Yes}\PY{l+s+s1}{\PYZsq{}}\PY{p}{]}\PY{p}{]}\PY{o}{.}\PY{n}{corr}\PY{p}{(}\PY{p}{)}
\end{Verbatim}


\begin{Verbatim}[commandchars=\\\{\}]
{\color{outcolor}Out[{\color{outcolor}65}]:}               SMS\_received       Yes
         SMS\_received      1.000000  0.128601
         Yes               0.128601  1.000000
\end{Verbatim}
            
    The table shows that there is no correlation between receiving a text
and attendance which is surprising indeed: as the author expected this
to show some correlation. In some countries (eg. the UK), text reminders
have worked to increase attendance. However, in other countries such as
the Netherlands, the same experiment failed to make any difference to
attendance rates.

    \begin{Verbatim}[commandchars=\\\{\}]
{\color{incolor}In [{\color{incolor}66}]:} \PY{c+c1}{\PYZsh{} Is chronic illness correlated to no\PYZhy{}show? }
         \PY{n}{nosho}\PY{p}{[}\PY{p}{[}\PY{l+s+s1}{\PYZsq{}}\PY{l+s+s1}{Hipertension}\PY{l+s+s1}{\PYZsq{}}\PY{p}{,}\PY{l+s+s1}{\PYZsq{}}\PY{l+s+s1}{Diabetes}\PY{l+s+s1}{\PYZsq{}}\PY{p}{,}\PY{l+s+s1}{\PYZsq{}}\PY{l+s+s1}{Alcoholism}\PY{l+s+s1}{\PYZsq{}}\PY{p}{,}\PY{l+s+s1}{\PYZsq{}}\PY{l+s+s1}{Handcap}\PY{l+s+s1}{\PYZsq{}}\PY{p}{,}\PY{l+s+s1}{\PYZsq{}}\PY{l+s+s1}{Yes}\PY{l+s+s1}{\PYZsq{}}\PY{p}{]}\PY{p}{]}\PY{o}{.}\PY{n}{corr}\PY{p}{(}\PY{p}{)}
\end{Verbatim}


\begin{Verbatim}[commandchars=\\\{\}]
{\color{outcolor}Out[{\color{outcolor}66}]:}               Hipertension  Diabetes  Alcoholism   Handcap       Yes
         Hipertension      1.000000  0.432652    0.087614  0.080345 -0.035752
         Diabetes          0.432652  1.000000    0.018704  0.058284 -0.014210
         Alcoholism        0.087614  0.018704    1.000000  0.002339 -0.000644
         Handcap           0.080345  0.058284    0.002339  1.000000 -0.006024
         Yes              -0.035752 -0.014210   -0.000644 -0.006024  1.000000
\end{Verbatim}
            
    Admittedly, the above table is included for thoroughness. The author did
not expect to see any correlations here. Chronic illness may affect the
speed in recieving an appointment but it is unlikely to affect
attendance rates.

    \begin{Verbatim}[commandchars=\\\{\}]
{\color{incolor}In [{\color{incolor}67}]:} \PY{c+c1}{\PYZsh{} Does being in the Bolsa Social mean they were more likely to attend?}
         \PY{n}{nosho}\PY{p}{[}\PY{p}{[}\PY{l+s+s1}{\PYZsq{}}\PY{l+s+s1}{Scholarship}\PY{l+s+s1}{\PYZsq{}}\PY{p}{,} \PY{l+s+s1}{\PYZsq{}}\PY{l+s+s1}{No}\PY{l+s+s1}{\PYZsq{}}\PY{p}{]}\PY{p}{]}\PY{o}{.}\PY{n}{corr}\PY{p}{(}\PY{p}{)}
\end{Verbatim}


\begin{Verbatim}[commandchars=\\\{\}]
{\color{outcolor}Out[{\color{outcolor}67}]:}              Scholarship        No
         Scholarship     1.000000 -0.029457
         No             -0.029457  1.000000
\end{Verbatim}
            
    The table above shows that scholarship has no correlation to the
attendance rates either. the author cannot draw any conculsions from
this because the context surrounding data collection is unknown: where
members of the Bolsa Familia program randomly picked? Was this an
experiment to see who would attend: those with financial aid or those
without?

    \begin{Verbatim}[commandchars=\\\{\}]
{\color{incolor}In [{\color{incolor}81}]:} \PY{c+c1}{\PYZsh{}Neighborhoods should be interesting as well}
         \PY{n}{neighborhoods} \PY{o}{=} \PY{n}{pd}\PY{o}{.}\PY{n}{crosstab}\PY{p}{(}\PY{n}{index} \PY{o}{=} \PY{n}{nosho}\PY{p}{[}\PY{l+s+s1}{\PYZsq{}}\PY{l+s+s1}{Neighbourhood}\PY{l+s+s1}{\PYZsq{}}\PY{p}{]}\PY{p}{,} \PY{n}{columns} \PY{o}{=} \PY{l+s+s1}{\PYZsq{}}\PY{l+s+s1}{count}\PY{l+s+s1}{\PYZsq{}}\PY{p}{)}
         \PY{n}{neighborhoods}\PY{o}{.}\PY{n}{plot}\PY{p}{(}\PY{n}{kind} \PY{o}{=}\PY{l+s+s1}{\PYZsq{}}\PY{l+s+s1}{bar}\PY{l+s+s1}{\PYZsq{}}\PY{p}{,} \PY{n}{figsize}\PY{o}{=}\PY{p}{(}\PY{l+m+mi}{20}\PY{p}{,}\PY{l+m+mi}{20}\PY{p}{)}\PY{p}{)}
         \PY{n}{plt}\PY{o}{.}\PY{n}{title}\PY{p}{(}\PY{l+s+s1}{\PYZsq{}}\PY{l+s+s1}{Figure 6: Neighborhoods distribution}\PY{l+s+s1}{\PYZsq{}}\PY{p}{,} \PY{n}{fontsize} \PY{o}{=} \PY{l+m+mi}{16}\PY{p}{)}
         \PY{n}{plt}\PY{o}{.}\PY{n}{ylabel}\PY{p}{(}\PY{l+s+s1}{\PYZsq{}}\PY{l+s+s1}{total number of observations}\PY{l+s+s1}{\PYZsq{}}\PY{p}{)}
\end{Verbatim}


\begin{Verbatim}[commandchars=\\\{\}]
{\color{outcolor}Out[{\color{outcolor}81}]:} <matplotlib.axes.\_subplots.AxesSubplot at 0x10488eb00>
\end{Verbatim}
            
\begin{Verbatim}[commandchars=\\\{\}]
{\color{outcolor}Out[{\color{outcolor}81}]:} <matplotlib.text.Text at 0x10db37748>
\end{Verbatim}
            
\begin{Verbatim}[commandchars=\\\{\}]
{\color{outcolor}Out[{\color{outcolor}81}]:} <matplotlib.text.Text at 0x10f6d7828>
\end{Verbatim}
            
    \begin{center}
    \adjustimage{max size={0.9\linewidth}{0.9\paperheight}}{output_51_3.png}
    \end{center}
    { \hspace*{\fill} \\}
    
    The figure above shows the dsitribution of observations per
neighborhood. The neighborhood of Jardim Camburi stands out the most.
Without any background information on the neighborhood and the
historical context when the survey was conducted, the author cannot draw
any interesting conclusions. Other than that, the most number of people
came from this area. However, as there is a wealth of information per
neighborhood. It would be interesting to see how the no-shows have been
distribtuted across the city

    \begin{Verbatim}[commandchars=\\\{\}]
{\color{incolor}In [{\color{incolor}83}]:} \PY{c+c1}{\PYZsh{} Let\PYZsq{}s see how no\PYZhy{}shows are distributed across the city}
         \PY{n}{nosho\PYZus{}hood} \PY{o}{=} \PY{n}{nosho}\PY{o}{.}\PY{n}{groupby}\PY{p}{(}\PY{l+s+s1}{\PYZsq{}}\PY{l+s+s1}{Neighbourhood}\PY{l+s+s1}{\PYZsq{}}\PY{p}{)}
         \PY{n}{nosho\PYZus{}hood}\PY{o}{.}\PY{n}{describe}\PY{p}{(}\PY{p}{)}\PY{o}{.}\PY{n}{head}\PY{p}{(}\PY{p}{)}
\end{Verbatim}


\begin{Verbatim}[commandchars=\\\{\}]
{\color{outcolor}Out[{\color{outcolor}83}]:}                         Age                                                  \textbackslash{}
                               count       mean        std   min    25\%   50\%    75\%   
         Neighbourhood                                                                 
         AEROPORTO               8.0  53.125000  10.934056  36.0  47.50  58.0  59.75   
         ANDORINHAS           2136.0  35.813202  22.406783   0.0  17.00  35.0  52.00   
         ANTÔNIO HONÓRIO       271.0  36.845018  23.606299   0.0  17.00  33.0  53.00   
         ARIOVALDO FAVALESSA   280.0  32.939286  23.695346   0.0  12.75  29.0  52.00   
         BARRO VERMELHO        422.0  45.109005  20.092481   0.0  30.00  48.5  60.00   
         
                                    Alcoholism           {\ldots}  Scholarship          Yes  \textbackslash{}
                                max      count      mean {\ldots}          75\%  max   count   
         Neighbourhood                                   {\ldots}                             
         AEROPORTO             64.0        8.0  0.000000 {\ldots}          0.0  0.0     8.0   
         ANDORINHAS           115.0     2136.0  0.022004 {\ldots}          0.0  1.0  2136.0   
         ANTÔNIO HONÓRIO      100.0      271.0  0.000000 {\ldots}          0.0  1.0   271.0   
         ARIOVALDO FAVALESSA   97.0      280.0  0.050000 {\ldots}          0.0  1.0   280.0   
         BARRO VERMELHO        92.0      422.0  0.004739 {\ldots}          0.0  0.0   422.0   
         
                                                                           
                                  mean       std  min  25\%  50\%  75\%  max  
         Neighbourhood                                                     
         AEROPORTO            0.125000  0.353553  0.0  0.0  0.0  0.0  1.0  
         ANDORINHAS           0.221910  0.415628  0.0  0.0  0.0  0.0  1.0  
         ANTÔNIO HONÓRIO      0.184502  0.388611  0.0  0.0  0.0  0.0  1.0  
         ARIOVALDO FAVALESSA  0.221429  0.415952  0.0  0.0  0.0  0.0  1.0  
         BARRO VERMELHO       0.215640  0.411754  0.0  0.0  0.0  0.0  1.0  
         
         [5 rows x 104 columns]
\end{Verbatim}
            
    This clearly isn't going to work with the describe function on all the
columns.

    \begin{Verbatim}[commandchars=\\\{\}]
{\color{incolor}In [{\color{incolor}84}]:} \PY{c+c1}{\PYZsh{}Plot of neighborhood against the mean no\PYZhy{}shows}
         \PY{n}{nosho\PYZus{}hood}\PY{p}{[}\PY{l+s+s1}{\PYZsq{}}\PY{l+s+s1}{No}\PY{l+s+s1}{\PYZsq{}}\PY{p}{,}\PY{l+s+s1}{\PYZsq{}}\PY{l+s+s1}{Yes}\PY{l+s+s1}{\PYZsq{}}\PY{p}{]}\PY{o}{.}\PY{n}{mean}\PY{p}{(}\PY{p}{)}\PY{o}{.}\PY{n}{plot}\PY{p}{(}\PY{n}{kind} \PY{o}{=} \PY{l+s+s1}{\PYZsq{}}\PY{l+s+s1}{bar}\PY{l+s+s1}{\PYZsq{}}\PY{p}{,} \PY{n}{figsize}\PY{o}{=}\PY{p}{(}\PY{l+m+mi}{20}\PY{p}{,}\PY{l+m+mi}{20}\PY{p}{)}\PY{p}{,} \PY{n}{stacked} \PY{o}{=} \PY{k+kc}{True}\PY{p}{)}
         \PY{n}{plt}\PY{o}{.}\PY{n}{title}\PY{p}{(}\PY{l+s+s1}{\PYZsq{}}\PY{l+s+s1}{Figure 7: Attendance and cancellation per neighborhood}\PY{l+s+s1}{\PYZsq{}}\PY{p}{,} \PY{n}{fontsize} \PY{o}{=} \PY{l+m+mi}{16}\PY{p}{)}
         \PY{n}{plt}\PY{o}{.}\PY{n}{ylabel}\PY{p}{(}\PY{l+s+s1}{\PYZsq{}}\PY{l+s+s1}{Attendance rates}\PY{l+s+s1}{\PYZsq{}}\PY{p}{)}
\end{Verbatim}


\begin{Verbatim}[commandchars=\\\{\}]
{\color{outcolor}Out[{\color{outcolor}84}]:} <matplotlib.axes.\_subplots.AxesSubplot at 0x103e4cf98>
\end{Verbatim}
            
\begin{Verbatim}[commandchars=\\\{\}]
{\color{outcolor}Out[{\color{outcolor}84}]:} <matplotlib.text.Text at 0x10dd27390>
\end{Verbatim}
            
\begin{Verbatim}[commandchars=\\\{\}]
{\color{outcolor}Out[{\color{outcolor}84}]:} <matplotlib.text.Text at 0x10f6c9160>
\end{Verbatim}
            
    \begin{center}
    \adjustimage{max size={0.9\linewidth}{0.9\paperheight}}{output_55_3.png}
    \end{center}
    { \hspace*{\fill} \\}
    
    The stacked bar graph above shows attendance rates across the city.
According to this graph above, the neighborhoods look evenly distributed
when it comes to attendance rates. Of course, there are two exceptions:
Ilhas Oceanias has 0 attendance while Parque Idnustrial has full
attendance. These are not errors. Both the neighborhoods only have one
sample each.

    \subsubsection{Is there any relation between waiting time and
attendance?}\label{is-there-any-relation-between-waiting-time-and-attendance}

Perhaps the longer someone waits, the more likely the attendance is
missed? This variable was created because it was, in fact, in the
original version of the data (version 1: "Awaiting time") and for some
reason removed in subsequent versions. However, it could be interesting:
the hypothesis being that the longer the waiting time, the more
frustrated the patient becomes or more likely they are to forget their
appointment.

    \begin{Verbatim}[commandchars=\\\{\}]
{\color{incolor}In [{\color{incolor}86}]:} \PY{c+c1}{\PYZsh{} Duration between scheduled day and apointment day}
         \PY{n}{nosho}\PY{p}{[}\PY{l+s+s1}{\PYZsq{}}\PY{l+s+s1}{Waiting\PYZus{}time}\PY{l+s+s1}{\PYZsq{}}\PY{p}{]} \PY{o}{=} \PY{n}{nosho}\PY{p}{[}\PY{l+s+s1}{\PYZsq{}}\PY{l+s+s1}{AppointmentDay}\PY{l+s+s1}{\PYZsq{}}\PY{p}{]}\PY{o}{.}\PY{n}{sub}\PY{p}{(}\PY{n}{nosho}\PY{p}{[}\PY{l+s+s1}{\PYZsq{}}\PY{l+s+s1}{ScheduledDay}\PY{l+s+s1}{\PYZsq{}}\PY{p}{]}\PY{p}{,} \PY{n}{axis}\PY{o}{=}\PY{l+m+mi}{0}\PY{p}{)}
         \PY{n}{nosho}\PY{o}{.}\PY{n}{Waiting\PYZus{}time}\PY{o}{.}\PY{n}{head}\PY{p}{(}\PY{l+m+mi}{10}\PY{p}{)}
\end{Verbatim}


\begin{Verbatim}[commandchars=\\\{\}]
{\color{outcolor}Out[{\color{outcolor}86}]:} 0     2 days 13:03:19
         1   -1 days +09:37:02
         2   -1 days +14:47:31
         3     2 days 16:46:24
         4     6 days 16:40:03
         5     2 days 12:01:42
         6   -1 days +14:17:53
         7    20 days 10:22:48
         8    18 days 16:36:30
         9    26 days 12:37:45
         Name: Waiting\_time, dtype: timedelta64[ns]
\end{Verbatim}
            
    \begin{Verbatim}[commandchars=\\\{\}]
{\color{incolor}In [{\color{incolor}87}]:} \PY{c+c1}{\PYZsh{} Convert the datetime object to integer for easier manipulation}
         \PY{n}{nosho}\PY{o}{.}\PY{n}{Waiting\PYZus{}time} \PY{o}{=} \PY{n}{nosho}\PY{o}{.}\PY{n}{Waiting\PYZus{}time}\PY{o}{.}\PY{n}{dt}\PY{o}{.}\PY{n}{days}
         \PY{n}{nosho}\PY{o}{.}\PY{n}{Waiting\PYZus{}time}\PY{o}{.}\PY{n}{head}\PY{p}{(}\PY{l+m+mi}{10}\PY{p}{)}
\end{Verbatim}


\begin{Verbatim}[commandchars=\\\{\}]
{\color{outcolor}Out[{\color{outcolor}87}]:} 0     2
         1    -1
         2    -1
         3     2
         4     6
         5     2
         6    -1
         7    20
         8    18
         9    26
         Name: Waiting\_time, dtype: int64
\end{Verbatim}
            
    \begin{Verbatim}[commandchars=\\\{\}]
{\color{incolor}In [{\color{incolor}88}]:} \PY{n}{nosho}\PY{o}{.}\PY{n}{Waiting\PYZus{}time}\PY{o}{.}\PY{n}{describe}\PY{p}{(}\PY{p}{)}
         \PY{n}{nosho}\PY{o}{.}\PY{n}{boxplot}\PY{p}{(}\PY{l+s+s1}{\PYZsq{}}\PY{l+s+s1}{Waiting\PYZus{}time}\PY{l+s+s1}{\PYZsq{}}\PY{p}{,} \PY{n}{vert} \PY{o}{=} \PY{k+kc}{False}\PY{p}{)}
         \PY{n}{plt}\PY{o}{.}\PY{n}{title}\PY{p}{(}\PY{l+s+s1}{\PYZsq{}}\PY{l+s+s1}{Figure 8: Waiting time box plot}\PY{l+s+s1}{\PYZsq{}}\PY{p}{,} \PY{n}{fontsize} \PY{o}{=} \PY{l+m+mi}{16}\PY{p}{)}
         \PY{n}{plt}\PY{o}{.}\PY{n}{xlabel}\PY{p}{(}\PY{l+s+s1}{\PYZsq{}}\PY{l+s+s1}{Days}\PY{l+s+s1}{\PYZsq{}}\PY{p}{)}
\end{Verbatim}


\begin{Verbatim}[commandchars=\\\{\}]
{\color{outcolor}Out[{\color{outcolor}88}]:} count    109193.000000
         mean          9.065746
         std          15.200463
         min          -7.000000
         25\%          -1.000000
         50\%           3.000000
         75\%          13.000000
         max         178.000000
         Name: Waiting\_time, dtype: float64
\end{Verbatim}
            
\begin{Verbatim}[commandchars=\\\{\}]
{\color{outcolor}Out[{\color{outcolor}88}]:} <matplotlib.axes.\_subplots.AxesSubplot at 0x10ea3a358>
\end{Verbatim}
            
\begin{Verbatim}[commandchars=\\\{\}]
{\color{outcolor}Out[{\color{outcolor}88}]:} <matplotlib.text.Text at 0x110026748>
\end{Verbatim}
            
\begin{Verbatim}[commandchars=\\\{\}]
{\color{outcolor}Out[{\color{outcolor}88}]:} <matplotlib.text.Text at 0x10fb33908>
\end{Verbatim}
            
    \begin{center}
    \adjustimage{max size={0.9\linewidth}{0.9\paperheight}}{output_60_4.png}
    \end{center}
    { \hspace*{\fill} \\}
    
    According to the figure above, most people only had to wait 9 days for
an appointment which is actually very efficient. However, it shoudl be
noted again that since th edata is skewed to the right, the sample size
is much greater for the shorter duration.

    \begin{Verbatim}[commandchars=\\\{\}]
{\color{incolor}In [{\color{incolor}89}]:} \PY{n}{nosho}\PY{o}{.}\PY{n}{Waiting\PYZus{}time}\PY{o}{.}\PY{n}{hist}\PY{p}{(}\PY{p}{)}
         \PY{n}{plt}\PY{o}{.}\PY{n}{ylabel}\PY{p}{(}\PY{l+s+s1}{\PYZsq{}}\PY{l+s+s1}{Observations}\PY{l+s+s1}{\PYZsq{}}\PY{p}{)}
         \PY{n}{plt}\PY{o}{.}\PY{n}{xlabel}\PY{p}{(}\PY{l+s+s1}{\PYZsq{}}\PY{l+s+s1}{Waiting time}\PY{l+s+s1}{\PYZsq{}}\PY{p}{)}
         \PY{n}{plt}\PY{o}{.}\PY{n}{title}\PY{p}{(}\PY{l+s+s1}{\PYZsq{}}\PY{l+s+s1}{Figure 9: Waiting time distribution}\PY{l+s+s1}{\PYZsq{}}\PY{p}{,} \PY{n}{fontsize} \PY{o}{=} \PY{l+m+mi}{16}\PY{p}{)}
\end{Verbatim}


\begin{Verbatim}[commandchars=\\\{\}]
{\color{outcolor}Out[{\color{outcolor}89}]:} <matplotlib.axes.\_subplots.AxesSubplot at 0x10ee40d68>
\end{Verbatim}
            
\begin{Verbatim}[commandchars=\\\{\}]
{\color{outcolor}Out[{\color{outcolor}89}]:} <matplotlib.text.Text at 0x10eb36b70>
\end{Verbatim}
            
\begin{Verbatim}[commandchars=\\\{\}]
{\color{outcolor}Out[{\color{outcolor}89}]:} <matplotlib.text.Text at 0x11001e390>
\end{Verbatim}
            
\begin{Verbatim}[commandchars=\\\{\}]
{\color{outcolor}Out[{\color{outcolor}89}]:} <matplotlib.text.Text at 0x10ea9c9b0>
\end{Verbatim}
            
    \begin{center}
    \adjustimage{max size={0.9\linewidth}{0.9\paperheight}}{output_62_4.png}
    \end{center}
    { \hspace*{\fill} \\}
    
    This is complementary to the boxplot figure above to show skewness.
Again, it shows a right skewed distribution which we need to take into
account before jumping to conclusions about impact of waiting time on
attendance rates

    \begin{Verbatim}[commandchars=\\\{\}]
{\color{incolor}In [{\color{incolor}92}]:} \PY{n}{nosho}\PY{p}{[}\PY{p}{[}\PY{l+s+s1}{\PYZsq{}}\PY{l+s+s1}{Waiting\PYZus{}time}\PY{l+s+s1}{\PYZsq{}}\PY{p}{,}\PY{l+s+s1}{\PYZsq{}}\PY{l+s+s1}{Yes}\PY{l+s+s1}{\PYZsq{}}\PY{p}{]}\PY{p}{]}\PY{o}{.}\PY{n}{corr}\PY{p}{(}\PY{p}{)}
\end{Verbatim}


\begin{Verbatim}[commandchars=\\\{\}]
{\color{outcolor}Out[{\color{outcolor}92}]:}               Waiting\_time       Yes
         Waiting\_time      1.000000  0.185291
         Yes               0.185291  1.000000
\end{Verbatim}
            
    The table above shows that there is no correlation between waiting time
and attendance. And even if there were, it would be hard to drawn a
conclusion because the distribution is not even.

    \begin{Verbatim}[commandchars=\\\{\}]
{\color{incolor}In [{\color{incolor}91}]:} \PY{c+c1}{\PYZsh{} does waiting time impact attendance? Even though the correlation table doesn\PYZsq{}t seem to point to anything conclusive}
         
         \PY{n}{sns}\PY{o}{.}\PY{n}{set}\PY{p}{(}\PY{n}{style}\PY{o}{=}\PY{l+s+s2}{\PYZdq{}}\PY{l+s+s2}{ticks}\PY{l+s+s2}{\PYZdq{}}\PY{p}{,} \PY{n}{color\PYZus{}codes}\PY{o}{=}\PY{k+kc}{True}\PY{p}{)}
         \PY{n}{sns}\PY{o}{.}\PY{n}{pairplot}\PY{p}{(}\PY{n}{nosho}\PY{p}{,} \PY{n+nb}{vars}\PY{o}{=}\PY{p}{[}\PY{l+s+s2}{\PYZdq{}}\PY{l+s+s2}{Age}\PY{l+s+s2}{\PYZdq{}}\PY{p}{,} \PY{l+s+s2}{\PYZdq{}}\PY{l+s+s2}{Waiting\PYZus{}time}\PY{l+s+s2}{\PYZdq{}}\PY{p}{]}\PY{p}{,} \PY{n}{hue}\PY{o}{=}\PY{l+s+s1}{\PYZsq{}}\PY{l+s+s1}{No\PYZhy{}show}\PY{l+s+s1}{\PYZsq{}}\PY{p}{)}
         \PY{n}{plt}\PY{o}{.}\PY{n}{title}\PY{p}{(}\PY{l+s+s1}{\PYZsq{}}\PY{l+s+s1}{Figure 9: Waiting time and age pair plots}\PY{l+s+s1}{\PYZsq{}}\PY{p}{,} \PY{n}{fontsize} \PY{o}{=} \PY{l+m+mi}{16}\PY{p}{)}
\end{Verbatim}


\begin{Verbatim}[commandchars=\\\{\}]
{\color{outcolor}Out[{\color{outcolor}91}]:} <seaborn.axisgrid.PairGrid at 0x10dd7c7f0>
\end{Verbatim}
            
\begin{Verbatim}[commandchars=\\\{\}]
{\color{outcolor}Out[{\color{outcolor}91}]:} <matplotlib.text.Text at 0x111c562e8>
\end{Verbatim}
            
    \begin{center}
    \adjustimage{max size={0.9\linewidth}{0.9\paperheight}}{output_66_2.png}
    \end{center}
    { \hspace*{\fill} \\}
    
    The figure above shows that waiting time and age are not correlated to
attendance rates. The stacked bar plots clearly show that the no
attendance rates (in green) although higher in the bottom left, appear
just as much across the distribution. It must be noted again that the
high peak in the bottom left is due to the fact that the sample size for
shorter durations are larger than larger durations. The only conclusion
that can be drawn from this is that amongst the population, the majority
of the sample waited around 9 days. But this conclusion cannot be
generalised for the general population until context of the experiment
is known.

     \#\# Conclusions

    What factors are important for us to know in order to predict if a
patient will show up for their scheduled appointment?

Part 1: What factors are highly correlated to outcome?

Before drawing any conclusions, it is important to note that the data is
very skewed and biased. There are 70\% "shows" compared to "no-shows"
(Figure 4). This may affect correlation between outcome and variables
such as Age. If the data is skewed it invariably means that the mean and
median will be off-centre and this will affect the Pearson coefficient.
So correlation analysis will not hold up. this will also affect
prediction analysis later on, although this can be accounted for in
predictive algorithms and models.

Some key variables are also skewed in the dataset: Age and outcome, for
example. This may explain why Age is not correlated? On the other hand
it throws into doubt the method used for correlation. Perhaps another
correlation method (such as Spearman) should be used for nominal and
continuous variables? The author is aware that the CHI-sq tests and
ANOVA are probably better test to use with categorical data however it
is outside the scope of this project.

More satisfying is the analysis done by grouping the ages (Figure 3). It
is possible that younger ages (0-18 and 30-45) tend not to show up for
their appointments. As the ages increase, the no-shows dissapear.
However, it should also be noted that the sample size for older ages is
much smaller than for 0-18 and 18-30. So there is a limitation here in
drawing conclusions.

Other variables have shown no correlation, and therefore further
analysis nor figures have been produced. For example, there doesn't seem
to be any suggestion that gender impacted attendance.

The neighborhoods data, although interesintg has not thrown anything
interesting. Looking at Figure 7, the distribution looks quite even
across the beighborhoods. Nothing (apart from the two neighborhoods that
have only one attendance and one cancellation) stand out. It is a pity
that there is no socio-economic data within the neighborhoods. A future
analysis could be neighborhoods and membership in the program. This may
provide some light on the economic status of the neighborhoods. Howeve,r
it will not shed any further light on whether a patient will show up for
their appointment or not.

Part 2: Is there any relationship between waiting time and attendance?

The simple answer is "no". From the analysis performed, there is no
obvious relationship. the mean waiting time is actually 9 days which
from the author's experience working within the healthcare industry is
very efficient. however, both no-shows and shows have similar
distributions over waiting time. this can be seen in the pairplots.

\subsection{Final conclusions}\label{final-conclusions}

The author believes that the wrong question was posed for this dataset.
It should not (and possibly cannot due to its 70\% bias) be used to
predict attendance rates. More interesting variables may have helped:
income and education status of patient, reason for appointment (you
cannot assume from illness what the reason for the appointment is).
Although, you could infer that, the more serious the illness and reason
for appointment, the shorter the waiting time.

A more interesting question would have been "What neighborhood have the
highest number of chronically ill patients?" or givne more
socio-economic data per neighborhood, "What impact do income levels have
with hospital attendance". These questions do not ask for a predictive
model but can be nevertheless insightful in the way people use
healthcare.


    % Add a bibliography block to the postdoc
    
    
    
    \end{document}
